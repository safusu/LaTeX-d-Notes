\documentclass[10pt,preprint]{article}
\usepackage[latin1]{inputenc}
\usepackage{amsmath}
\usepackage{amsfonts}
\usepackage{amssymb}
\usepackage{mathtools}
\usepackage{fullpage}

\newcommand{\pt}{\propto}
\newcommand{\rp}{\right)}
\newcommand{\lp}{\left(}

\begin{document}

\begin{center}
\textbf{\begin{huge} September 20, 2011\end{huge}}\\
\begin{large}Jeren Suzuki\end{large}
\end{center}

\section{Convection Continued}

\begin{align*}
a = |N|^2 \delta r\\
|N|^2 &= \frac{g}{cp} \Bigl\lvert \frac{ds}{dr} \Bigl\lvert\\
&= \frac{g}{H} \Bigl\lvert \frac{H}{c_p} \frac{ds}{dr} \Bigl\lvert\\
v_c^2 = a \delta r = |N|^2 \delta r^2\\
\delta r \equiv \alpha H\\
v_c = \alpha c_s \Bigl\lvert \frac{H}{c_p} \frac{ds}{dr} \Bigl\lvert^{1/2}\\
F = \frac{1}{2}\rho v_c^3 = \frac{1}{2} \rho \alpha ^3 c_s^3 \Bigl\lvert \frac{H}{c_p} \frac{ds}{dr} \Bigl\lvert ^{3/2}
\end{align}

We wanted to find the $F_r = -\frac{4}{3} \frac{caT^3}{\kappa \rho}\frac{dT}{dr}$ equivalent for convection. $F = \frac{1}{2}\rho v_c^3 = \frac{1}{2} \rho \alpha ^3 c_s^3 \Bigl\lvert \frac{H}{c_p} \frac{ds}{dr} \Bigl\lvert ^{3/2}$ gives the $v_c$ and $\frac{ds}{dr}$ given the flux. \\

$\Bigl\lvert \frac{H}{c_p} \frac{ds}{dr} \Bigl\lvert \sim 10^{-6}$, $s \sim c_p$, so $\frac{\Delta s }{s} \sim 10^{-6}$ on a length scale $~H$. Ergo, $s$ = constant in the convection zone. This replaces $F_r = -\frac{4}{3} \frac{caT^3}{\kappa \rho}\frac{dT}{dr}$. Let's assume $P \pt \rho^\gamma$ \& $\frac{dP}{dr} = -\rho \frac{GM_r}{r^2}$. 

\begin{align*}
\frac{dM_r}{dr} = 4 \pi r^2 \rho\\
\frac{d}{dr} \lp \frac{r^2}{\rho} \frac{dP}{dr} = -\rho GM_r \rp\\
\frac{d}{dr} \lp \frac{r^2}{\rho	} \frac{dP}{dr} \rp = -4\pi r^2 G\rho
\end{align}

But... if $P = K \rho^\gamma$, $\frac{dP}{dr} = \gamma K \rho^{\gamma-1} \frac{d\rho}{dr}$! These kinds of models are called:

\subsection{Polytropic Models}

\begin{align*}
P &= K\rho^\gamma\\
&= K \rho^{1 + 1/n}, \gamma = 1 + \frac{1}{n}, \text{ where n is the polytropic index}
\end{align}

\begin{align*}
\theta = \lp \frac{\rho}{\rho_c} \rp ^{1/n}~,\rho_c = \rho(r=0)\\
\zeta = \frac{r}{a}~, a = \sqrt{\frac{(n+1) K \rho_c ^{\frac{1}{n} -1}}{4 \pi G}}~,[a] = \text{ cm}
\end{align}

\begin{align*}
\boxed{\frac{1}{\zeta} \cdot \frac{d}{d\zeta} \lp  \zeta^2 \frac{d\theta}{d\zeta} \rp = -\theta^n}
\end{align}

Let's look at the properties of a fully convective star of low mass. Low mass $\rightarrow$ low $T$ $\rightarrow$ high $\kappa$. 

\subsection{$M_* < \frac{1}{3} M_\odot$ on MS}

For stars with photons carrying the energy out, $L \pt M^3$ if $\sigma = \sigma_T$ for fully convective stars, $L = 4 \pi R^2 F_c$, where $F_c = \rho v_c^3 \pt \Bigl\lvert \frac{ds}{dr} \Bigl\lvert^{3/2}$. Let's look at the surface where photons are carrying the energy out. 

\begin{align*}
\text{s = constant}\\
P \pt \rho^{5/3} \pt T^{5/2}\\
\rho T \pt \rho^{5/3}, T \pt \rho^{2/3}\\
\frac{P_c}{P_{photons}} = \lp \frac{T_c}{T_{eff}} \rp ^{5/2},\text{ now use V.T. to relate $T_c$ and $M$ \& $R$.}
\end{align}
\begin{center}
{\Huge BUT}
\end{center}
We know that this is a $n=3/2$ polytrope so $kT_c = .54 \frac{GM\mu m_p}{R}$



\end{document}