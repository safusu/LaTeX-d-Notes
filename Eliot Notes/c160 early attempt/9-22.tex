\documentclass[10pt,preprint]{article}
\usepackage[latin1]{inputenc}
\usepackage{amsmath}
\usepackage{amsfonts}
\usepackage{amssymb}
\usepackage{mathtools}
\usepackage{fullpage}

\newcommand{\pt}{\propto}
\newcommand{\rp}{\right)}
\newcommand{\lp}{\left(}

\begin{document}

\begin{center}
\textbf{\begin{huge} September 22, 2011\end{huge}}\\
\begin{large}Jeren Suzuki\end{large}
\end{center}

\section{Star Formation}

Gas in galaxies comes in multiple "phases". It's still a gas, just a broad range with particular characteristics. They have different $\rho$ \& $T$ with comparable $P$. Hot, low $\rho$ gas is mostly in the form of Hot ISM. Stars form from \textbf{cold molecular clouds}. What are the conditions for a cold molecular gas cloud to collapse? 

\begin{align*}
|U| &\geq |K|\\
\frac{GM}{R^2} &\gtrsim \Bigl\lvert \frac{dP}{dr} \Bigl\lvert\\
\text{self gravity of cloud} &\gtrsim \frac{3}{2}NkT\\
&\approx \frac{M}{m_p}kT
\end{align}

If $M \gtrsim \frac{RkT}{Gm_p}$, then it will collapse. 

\begin{align*}
\rho &\approx \frac{M}{R^3}\\
R &\sim \lp \frac{M}{\rho} \rp^{1/3}\\
M^{2/3} &\gtrsim \frac{kT}{Gm_p \rho^{1/3}}\\
\Aboxed{M &\geq \lp \frac{k}{Gm_p} \rp^{3/2} \frac{T^{3/2}}{\sqrt{\rho}}}~, \text{ Jeans Mass}\\
\frac{GM^2}{R} &\geq \frac{MkT}{m_p} \\
\frac{GM^2}{R} &\geq c_s\\
\test{if} \frac{1}{\sqrt{G\rho}} &\leq \frac{R}{c_s}~,\text{ then $t_{FF} < t_{sound}$, and it will collapse}
\end{align}

Stars are more prone to collapse if they have lower $T$ and higher $\rho$. Stars form from cold molecular clouds because they are the most unstable.

\begin{align*}
M_J &\approx 50 M_\odot \frac{\lp \frac{T}{10K}\rp^{3/2}}{\lp \frac{\text{n}}{100 \text{ cm}^{-3}} \rp^{1/2}}\\
R_J &= \lp \frac{M_J}{\rho} \rp^{1/3}\\
&\approx 3 \text{ pc} \frac{(T/10K)^{1/2}}{(n/100\text{ cm}^{-3})^{1/2}}
\end{align}

If a star has $M>M_J$ and $R<R_J$, then it will collapse. The collapse time is $\sim \frac{1}{\sqrt{G\rho}} \sim 10\text{ Myr} \lp \frac{n}{100 \text{ cm}^{-3}} \rp^{-1/2}$. Why don't we have tons of $50M_\odot$ stars? In reality, most of them are roughly $0.3M_\odot$. 

\begin{align*}
\rho a &= -\frac{dP}{dr}- \rho \frac{GM}{R^2}\\
&\sim \frac{P}{R} - \frac{GM^2}{R^5}\\
&\pt \frac{nT}{R}\\
&\pt \frac{MT}{R^4}
\end{align}

\section{Actual Collapse}

Initially, the gas cools rapidly and since photons easily escape cloud, $T \sim$ roughly constant, isothermal at around $10K$. $P \pt \frac{M}{R^4}$, \& gravity $\pt \frac{M^2}{R^5}$. As radius decreases,


\end{document}