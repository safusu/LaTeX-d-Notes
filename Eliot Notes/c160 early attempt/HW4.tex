\documentclass[10pt,a4paper]{article}
\usepackage[latin1]{inputenc}
\usepackage{amsmath}
\usepackage{amsfonts}
\usepackage{amssymb}
\usepackage{fullpage}
\author{Jeren}
\title{HW \# 4}
\begin{document}
\large{\textbf{Problem 3a}}
\begin{align*}
\frac{dM}{dr} & = 4\pi r^2 \rho\\
\int dM & = \int 4 \pi r^2 \rho dr \\
M & = 4 \pi a^3 \rho_c \int \theta^n \xi^2 d \xi \\
\xi^2 \frac{d\theta}{d \xi} & = - \int \xi^2 \theta^n d\xi\\
M & = -4 \pi a^3 \rho_c \xi^2 \frac{d \theta}{d \xi}\\
\rho_c & = \frac{M}{4 \pi a^3 \xi^2 \frac{d \theta}{d \xi}}~, \text{set} ~ a=\frac{R}{\xi_R}~ \text{and evaluate}~ \xi ~\text{at} ~ R\\
\rho_c & = \frac{M}{4 \pi R^3 } \frac{\xi_R}{\frac{d \theta}{d \xi_R}}\\
\rho_c & = \frac{3M}{4 \pi R^3 } \frac{\xi_R}{3 \frac{d \theta}{d \xi_R}}\\
\rho_c & = \frac{3M}{4 \pi R^3 }a_n ~, \text{where} ~a_n = \frac{\xi_r}{3 \frac{d \theta}{d \xi_R}}~ \text{and is dimensionless}
\end{align*}

\large{\textbf{Problem 3b}}
\begin{align*}
P & =\kappa \rho ^\gamma\\
P_c &  = \kappa \rho_c^\gamma\\
& = \kappa \frac{\rho_c^\gamma \rho_c^2}{\rho_c^2}\\
& = \kappa \rho_c ^{\gamma -2}\rho_c^2\\
& = \kappa \rho_c ^{1/n -1}\rho_c^2\\
& = \kappa \frac{n+1}{n+1} \frac{4 \pi G}{4 \pi G} \rho_c^{1/n -1} \rho_c^2\\
& = \frac{4 \pi G \rho_c^2}{n+1} \left(  \frac{\kappa(n+1)\rho_c^{1/n -1}}{4 \pi G} \right)\\
P_c & = \frac{4 \pi G \rho_c^2}{n+1}a^2 ~,\text{where}~ a^2 = \left(  \frac{\kappa(n+1)\rho_c^{1/n -1}}{4 \pi G} \right)
\end{align*}

\begin{align*}
P_c & = \frac{4 \pi G}{n+1}a^2 \left( \frac{3M}{4 \pi R^3} a_n\right)^2\\
& = \frac{9 G M^2 a^2 a_n^2}{(n+1)4\pi R^6}\\
& = \frac{9GM^2 \alpha^2 R^2 a_n^2}{(n+1)4\pi R^6}~, \text{where}~ a=\alpha R\\
& = \frac{GM^2}{R^4}\frac{9 \alpha^2 a_n^2}{(n+1)4 \pi}\\
P_c & = \frac{GM^2}{R^4}c_n~, \text{where} ~c_n=\frac{9 \alpha^2 a_n^2}{(n+1)4 \pi}
\end{align*}

\begin{align*}
\bar{\rho} &= \frac{M}{\frac{4 \pi}{3}R^3}\\
\bar{\rho}^{4/3} & = \left( \frac{M}{\frac{4 \pi}{3}R^3} \right)^{4/3}\\
R^4 & = \left( \frac{M}{\bar{\rho} \frac{4 \pi}{3}} \right)^{4/3}\\
\\
P_c & = \frac{GM^2}{R^4}c_n\\
& = GM^2{\left( \frac{M}{\bar{\rho} \frac{4 \pi}{3}} \right)^{-4/3}}c_n\\
& = GM^{2/3}\bar{\rho}^{4/3} \left( \frac{4 \pi}{3} \right)^{4/3} c_n\\
& = GM^{2/3}\bar{\rho}^{4/3} \left( \frac{4 \pi}{3} \right)^{4/3} c_n~, \bar{\rho} = \frac{\rho_c}{a_n}~,\\
& = GM^{2/3}\left(\frac{\rho_c}{a_n} \right)^{4/3} \left( \frac{4 \pi}{3} \right)^{4/3} c_n\\
& = GM^{2/3}\rho_c^{4/3}d_n~, \text{where}~ d_n = c_n \left( \frac{4 \pi}{3 a_n} \right)^{4/3}
\end{align*}

\large{\textbf{Problem 3c}}
\begin{flushleft}
\begin{align*}
\text{Set the two}~P_c ~\text{equations equal to each other}\\
d_n GM^{2/3}\rho_c^{4/3} & = \left( \frac{GM^2}{R^4} \right) c_n\\
d_n & = \frac{M^{4/3}}{R^4}c_n \rho_c^{-4/3}\\
& = \frac{M^{4/3}}{R^4}c_n \left( \frac{3M}{4 \pi R^3} a_n \right)^{-4/3}\\
& = \frac{M^{4/3}}{R^4}c_n \left(\frac{1}{\frac{3M}{4 \pi R^3}a_n} \right)^{4/3}\\
& = c_n \left(\frac{4 \pi}{3 a_n} \right)^{4/3}
\end{align*}
\end{flushleft}

\begin{align*}
d_n(n=3) = 11.05 \cdot \left(\frac{4 \pi}{3 \cdot 54.183} \right)^{4/3} &\approx .3639\\
d_n(n=1.5) = 0.77 \cdot \left(\frac{4 \pi}{3 \cdot 5.99} \right)^{4/3} &\approx .477
\end{align*}

\large{\textbf{Problem 3d}}

\begin{align*}
P_c & = \frac{\rho_c k T_c}{\mu m_p}\\
& = c_n \frac{GM^2}{R^4}~, \text{from class notes}\\
c_n \frac{GM^2}{R^4} & = \frac{\rho_c k T_c}{\mu m_p}~, \rho_c = \left( \frac{3M}{4 \pi R^3} \right) a_n\\
T_c & =\frac{\mu m_p c_n}{a_n} \frac{GM 4 \pi}{3kR}
\end{align*}

\large{\textbf{Problem 3e}}

\begin{align*}
\text{For n=3}\\
T_c & =\frac{\mu m_p c_n}{a_n} \frac{GM 4 \pi}{3kR}\\
& = \frac{.5 \cdot 1.67 \times10^{-24} \cdot 11.05}{54.183} \frac{6.67 \times 10^{-8}\cdot 2 \times 10^{33} \cdot 4\pi}{6.96 \times 10^{10} \cdot 1.38 \times 10^{-16}}\\
& = 2.97 \times10^7 K\\
\rho_c & = \frac{3M}{4 \pi R^3}a_n\\
& = \frac{3 \cdot 2 \times 10^{33}}{4 \pi (6.96 \times 10^{10})^3}54.183\\
& = 76.7~ \text{gm cm}^{-3}\\
P_c & = \frac{GM^2}{R^4}c_n\\
& = \frac{6.67 \times 10^{-8} \cdot (2 \times 10^{33})^2}{(6.96 \times 10^{10})^4}11.05\\
& = 1.25 \times10^{17} \text{dyne cm}^{-3}
\end{align*}

\begin{align*}
\text{For n=1.5}\\
T_c & =\frac{\mu m_p c_n}{a_n} \frac{GM 4 \pi}{3kR}\\
& = \frac{.5 \cdot 1.67 \times10^{-24} \cdot 0.77}{5.99} \frac{6.67 \times 10^{-8}\cdot 2 \times 10^{33} \cdot 4\pi}{6.96 \times 10^{10} \cdot 1.38 \times 10^{-16}}\\
& = 1.87 \times10^7 K\\
\rho_c & = \frac{3M}{4 \pi R^3}a_n\\
& = \frac{3 \cdot 2 \times 10^{33}}{4 \pi (6.96 \times 10^{10})^3}5.99\\
& = 8.48~ \text{gm cm}^{-3}\\
P_c & = \frac{GM^2}{R^4}c_n\\
& = \frac{6.67 \times 10^{-8} \cdot (2 \times 10^{33})^2}{(6.96 \times 10^{10})^4}0.77\\
& = 8.7 \times10^{15} \text{dyne cm}^{-3}
\end{align*}

I think the $n=3$ polytrope index better approximates the interior environment of the sun better than $n=1.5$ because a higher polytropic index corresponds to $\gamma = 4/3$, which is for a relativistic gas. The center of the sun is very dense and hot and energy is carried through radiation. \\

\large{\textbf{Problem 4}}

\begin{align*}
P_c = \frac{GM^2}{R^4}c_n\\
P_{ph} = \frac{GM}{R^2 \kappa_{ph}}\\
\left( \frac{P_c}{P_{ph}}\right)^{2/5} & = \frac{Tc}{T_{eff}}\\
T_{eff} & = T_c \left( \frac{P_{ph}}{P_c} \right)^{2/5}\\
& = T_c \left( \frac{GM}{R^2 \kappa_{ph}} \frac{R^4}{GM^2 c_n} \right)^{2/5}\\
& = T_c \left( \frac{R^2}{\kappa_{ph}M c_n} \right)^{2/5}\\
& = T_c \left( \frac{R^2}{\kappa_{ph}M c_n} \frac{\rho_{ph}}{\rho_{ph}}\right)^{2/5}\\
& = T_c \left( \frac{lR^2 \rho_{ph}}{M c_n}  \right)^{2/5}\\
\\
\rho_{ph} & = \bar{\rho}\\
& = \frac{M}{\frac{4 \pi R^3}{3}}\\
\\
T_{eff} & =T_c \left( \frac{3l}{R 4 \pi c_n} \right)^{2/5}\\
& =0.626  \cdot T_c\left(\frac{l}{R} \right) ^{2/5}~, c_n = \text{0.77}
\end{align*}

\end{document}