\documentclass[10pt,letterpaper,final]{book}
\usepackage[latin1]{inputenc}
\usepackage{amsmath}
\usepackage{amsfonts}
\usepackage{amssymb}
\usepackage{fullpage}
\usepackage{mathtools}
\pagestyle{fancy}
\author{Jeren}

\newcommand{\pt}{\propto}
\newcommand{\rp}{\right)}
\newcommand{\lp}{\left(}

\begin{document}

\frontmatter
\chapter*{\center \begin{LARGE}  Astronomy 160 - Stellar Physics (Fall 2011) \end{LARGE}}
\thispagestyle{empty}
\section*{\huge \center Eliot Quataert\footnote{Transcribed by Jeren Suzuki}} 

\begin{center}
\includegraphics[width=\textwidth /2]{latex-image-1.eps}
\end{center}

\tableofcontents
\include{index}
\include{titlepg}


\mainmatter

\chapter{Energy Transport}

\begin{center}
\textbf{\begin{huge} September 6, 2011\end{huge}}
\end{center}

\section{Energy Transport by Conduction}
\begin{figure}[!ht]
\centering
\includegraphics[width=\textwidth /2]{9-6/fig1.png}
\label{fig:1}
\end{figure}

Figure \ref{fig:1} shows the stratification levels of the inside of a star at a given height $x$ with a length $l$  either up or down. \\
\begin{list}{}{}
\item $T$ = temp
\item $u$ = Thermal Energy Density
\item $v$ = velocity
\item $l $= mean free path
\end{list}

The factor of $\frac{1}{6}$ is used because we multiply $\frac{1}{3}$ (from 3 degrees of translational freedom) with $\frac{1}{2}$ (from going up and down).

\begin{align}
F \equiv ~\text{net flux of energy} = \frac{1}{6}u(x-l)v - \frac{1}{6}u(x+l)v
\end{align}
Taylor expanding $u(x-l) = u(x) - \frac{du}{dx}l$, we get:

\begin{align}
F = -\frac{1}{3}vl\frac{du}{dx}~,~\text{or more generally,}~F=-\chi \nabla u
\end{align}

For a gas of charged particles,

\begin{align}
U &= n\frac{3}{2}kT\\
F &= -\frac{1}{3}vl \frac{dU}{dT}\frac{dT}{dx}~,~\text{where}~\frac{dU}{dT}~\text{is the specific heat}.\\
\Aboxed{F &= -\frac{1}{2}nklv \frac{dT}{dx}}
\end{align}

In the case of a charged particle moving past another charged particle, there is a characterized distance $b$ where the change in trajectory is significant. 

\begin{align}
\frac{q^2}{b} &\sim kT\\
b &\sim \frac{q^2}{kT}\\
\text{effective area }=\sigma &= \pi b^2\\
\Aboxed{\sigma &= \frac{\pi q^4}{(kT)^2}}
\end{align*}

\begin{align}
F &=-\frac{1}{2}nklv\frac{dT}{dx}~,v \sim v_{thm} = \sqrt{\frac{kT}{m}}\\
&\sim -\frac{1}{2} \frac{kv}{\sigma}\frac{dT}{dx}\\
&=-\chi \frac{dT}{dx}~,\text{where }\chi =\frac{1}{2}\frac{kv}{\sigma} \pt \frac{T^{5/2}}{\sqrt{m}}
\end{align}

Electrons move faster than protons so they transfer energy much more effectively. Electrons and protons have the same $\sigma$, but the difference in velocities is huge.

\subsection{How important is this energy transport in the sun?}

\begin{align}
F &= -\chi \frac{dT}{dx}\\
L &= 4 \pi r^2 F\\
&\sim 4 \pi R^2 \chi \frac{T_c}{R}~\text{(We're doing a poor man's derivative where $\frac{dT}{dx} = \frac{T_c}{R}$)}\\
L &\sim \frac{k^{7/2} T_c^{7/2} R}{q^4 \ln (\Lambda) \sqrt{m_e}}\\
&\sim 10^{-4} L_\odot \lp \frac{R}{R_\odot} \rp \lp \frac{T_c}{10^7\text{ K} } \rp^{7/2}
\end{align*}

Doing this, we get that $L_{conduction} \ll L_\odot$ and therefore electron conduction is unimportant to energy transport. 

\section{Radiation Transport of Energy}
\begin{list}{}{}
\item $U = aT^4$
\item $l = \frac{1}{n \sigma}$
\item $\sigma$ = cross section for photons to interact with matter, not with other photons
\end{list}

\begin{align}
F &=-\frac{1}{3}cl \frac{d}{dr}aT^4\\
&= -\frac{4}{3}claT^3 \frac{dT}{dr}\\
\Aboxed{&=-\frac{4}{3}\frac{caT^3}{\kappa \rho} \frac{dT}{dr}}
\end{align}

For an Ionized plasma with dominant photon-matter interactions through electron scattering (Thomson Scattering),

\begin{align}
m_e \bar{a} &= -e(\bar{E} + \frac{\bar{v}}{c} \times \bar{B})\\
|E| &= |B|\\
m_e\bar{a} &= -e\bar{E}\text{ since $ \frac{\bar{v}}{c} \times \bar{B}$ is small}\\
\bar{a} &= -\frac{e \bar{E}}{m_e}
\end{align}

\begin{align}
P &= \frac{2}{3}\frac{e^4}{c^3 m_e^2}|\bar{E}|^2\\
\sigma F &= P\\
\sigma \frac{c}{4\pi}|\bar{E}| &= \frac{2}{3}\frac{e^4}{c^3 m_e^2}|\bar{E}|^2\\
\Aboxed{\sigma_T &= \frac{8\pi}{3}\frac{e^4}{m_e^2 c^4} = \text{ Thomson cross-section or $e^-$-scattering cross-section}}
\end{align}

\begin{align}
\sigma_T &= \frac{8\pi}{3}r_c^2,~\text{ where $r_c$ is the classical radius of the $e^-$}\\
r_c &=\frac{e^2}{m_e c^2} \approx 2.8 \times 10^{-13} \text{ cm},
\end{align}
which should strike as odd since $e^-$ has no observed structure but still has an "effective radius".

%%%%%%%%%%%%%%%%%%%

\chapter{Conduction}

\begin{center}
\textbf{\begin{huge} September 8, 2011\end{huge}}
\end{center}

\section{Picking Up Where We Left Off}
\begin{align}
F &= -\frac{4}{3}\frac{caT^3}{\kappa \rho} \frac{dT}{dr} \leftarrow \text{ for radiative diffusion}\\
&= -\frac{1}{3}lc \frac{dU}{dr}
\end{align}

\begin{align}
\frac{dT}{dr}\sim \frac{T}{R}\\
kT \sim \frac{GM\mu m_p}{R}\\
L \sim \chi M^3\\
\sigma \sim \sigma_T\\
\chi = \frac{a (\mu m_p)^4 cG^4 \mu_e m_p}{\sigma_T k^4}\\
L &\sim 10^{35} \lp \frac{M}{M_\odot} \rp^3\text{ ergs s}^{-1}\\
&\sim L_\odot ~;e^- \text{ conduction $L \ll L_\odot$, so photons are dominating energy transport
\end{align}

$L$ is strictly \textit{NOT} dependent on fusion, it is more dependent on $\frac{dU}{dT}$ of the photons. Fusion creates $E$ and the photons but doesn't determine the rate of energy leaving.\\
$\chi$ is constant, but dependent on the composition of the star. 

\begin{align}
L \pt \mu^4 \mu_e M^3\\
H \rightarrow HE\\
\mu \rightarrow \mu \uparrow\\
L \rightarrow L \uparrow
\end{align}

$L$ was lower in the past. $L$ when the Earth forced was only 20\% of the current $L_\odot$. This brings about the problem that the Earth was supposedly a ball of ice during it's evolution. \\

\begin{align}
\frac{3}{2}nkT > aT^4\\
F &= -\frac{1}{3}lv \frac{dU}{dx}\\
&= -\chi \frac{dT}{dx},\chi = \frac{1}{3}lv \frac{dU}{dT}\\
\frac{F_{rad}}{F_{e^-}} &= \frac{-\chi_{rad}}{-\chi_{e^-}}, \frac{dT}{dx}\text{ is the same for both}\\
&\sim \frac{l_\gamma}{l_{e^-} }\frac{ c}{v_{e^-} }\frac{aT^4}{\frac{3}{2}nkT}
\end{align}

$ \frac{l_\gamma}{l_{e^-} } \ggg 1$, $\frac{ c}{v_{e^-}} \gg 1$, and $\frac{aT^4}{\frac{3}{2}nkT} \ll 1$.\\

\begin{align}
F = -\frac{4}{3}\frac{caT^4}{n \sigma} \frac{dT}{dr}
\end{align}

We want to find the time for thermal energy to leak out by photon diffusion.

\begin{align}
t_{KH} &= \frac{E}{L}\\
L &\sim 4 \pi R^2 \frac{4}{3}aT^3 \frac{1}{n\sigma}\frac{T}{R}\\
&= \frac{\frac{3}{2}nkT \cdot \dfrac{4}{3}\pi R^3}{ \lp \dfrac{4 \cdot 16 \pi aT^4R}{R3n\sigma} \rp}\\
&=\frac{\frac{3}{2}nkTR^2n\sigma}{4aT^4c}\\
&\sim \frac{nkT}{aT^4}\frac{R^2}{lc}\\
&\sim \frac{nkT}{aT^4}t_{RW}~,\text{ where $t_{RW}=\frac{R^2}{lc}=$ random walk time}
\end{align}

\begin{align}
<|D|^2>&=Nl^2~,\text{ where $N = $ number of steps}\\
<|D|^2>^{1/2} &= \text{ RMS Distance = typical distance a photon will find itself after $N$ scatterings}\\
&= \sqrt{N}l
\end{align}

What is $N$ so that the photon leaves the star?

\begin{align}
<|D|^2>^{1/2} &=R\\
N &\sim \lp \frac{R}{l} \rp^2\\
\frac{R}{l} \sim 10^{11},N\sim 10^{22}
\end{align}

How long does it take to get out? 

\begin{align}
Nt_{step} = t_{esc} = N\frac{l}{c} \sim \lp \frac{R}{l} \rp^2 \cdot \frac{l}{c} \sim \frac{R^2}{lc}\\
\sim 10^4 \text{ years for our sun}
\end{align}

If photons didn't bounce around, it would escape in 2 seconds. ($\nu_e$ can get out in about 2 seconds, $l_\nu$ must be greater than $R_\odot$) Also, the time it takes for heat to diffuse throughout a room is: $\frac{R^2}{lv_{thm}}$.

\begin{align}
F &= -\frac{4}{3}\frac{caT^4}{n \sigma} \frac{dT}{dr}\\
&= -lc \frac{d}{dr}\frac{1}{3}aT^4\\
F_r &= -lc\frac{d}{dr}P_{rad}~,\text{ we must be careful, $F$ and $L$ depend on $r$}\\
\frac{L_r}{4\pi r^2} &= -lc \frac{d}{dr}P_{rad}\\
-\frac{L_r}{4\pi lc r^2} &= \frac{d}{dr}P_{rad}\\
-\frac{L_r \kappa \rho}{4\pi c r^2} &= \frac{d}{dr}P_{rad}\\
\end{align}

We're interested in how $P_{rad}$ changes with $P_{tot}$. 

\begin{align}
\frac{dP_{rad}}{dP} = \frac{L_r \kappa}{4 \pi G M_r c} \equiv \frac{L_r}{L_{EDD}}
\end{align}

Roughly, if $P_{rad} \sim P_{tot}$, then $L_r \sim L_{EDD}$.

\subsection{Eddington Luminosity}

\begin{list}{}{}
\item $F_g = -\frac{GMm}{r^2}$
\item $F_{rad} = \frac{dp}{dt}$
\item $p_{proton} = \frac{E}{c} = \frac{h \nu}{c} = \frac{h}{\lambda}$
\end{list}

What is the total $p$ per unit time produced by the star? $\sum\limits{_i^\infty} p_i $ is too hard...

\begin{align}
p_{photon} = \frac{E}{c} = \frac{L}{c}\\
F_{Rad} = \frac{dp}{dt} = \frac{L}{c 4\pi r^2} \sigma
\end{align}

The Eddington Luminosity is where the radiation force equals the forge of gravity. The Eddington "Limit" is:

\begin{align}
L_{EDD} = \frac{4 \pi G Mc}{\sigma /m} = \frac{4 \pi G Mc}{\kappa}
\end{align}

If $L > L_{EDD}$, $F_{rad} > F_{grav}$, and material is "blown" out. 

\subsection{Fully Ionized H}

\begin{list}{}{}
\item $L_{EDD} = \frac{4 \pi GMc}{\kappa}$
\item $\kappa = \frac{\sigma_T}{m}$
\item $l = \frac{1}{n \sigma} = \frac{1}{n_e \sigma_T} = \frac{1}{\kappa \sigma}$
\item For fully ionized $H$, $\mu_e =1$
\item since photons are only interacting with $e^-$ and not $p$, $\mu = 1 = \mu_e$
\end{list}

\begin{align}
\kappa &= \frac{\sigma_T}{m}\\
&= 0.4\text{ cm}^2\text{ g}^{-1}
\end{align}

\begin{align}
L_{EDD} = 1.3 \times 10^{38} \lp \frac{M}{M_\odot} \rp \text{ ergs s}^{-1}
\end{align}

\begin{list}{}{}
\item If $L \sim L_{EDD}$, $F_{rad} \sim F_{grav}$ which results in the radiation force not being important in the sun. It's dominated by gas pressure.
\item As $M \uparrow, L_{EDD} \uparrow$ so higher $M$ means $P_{rad}$ becomes more important $\rightarrow$ it becomes the dominant force
\end{list}

%%%%%%%%%%%%%%%%%%%%%
\chapter{Convection}

\begin{center}
\textbf{\begin{huge} September 13, 2011\end{huge}}
\end{center}

\section{Convection}
Second Law of Thermodynamics: $TdS = dE + PdV$, which isn't all that useful for stars, really.
\begin{list}{}{}
\item $U = \frac{E}{\mu}$: Energy per unit mass
\item $s = \frac{S}{U}$: Entropy
\item $M =$ conserved, $l$ is small
\item $\rho = \frac{M}{V}, V = \frac{M}{\rho} \rightarrow \boxed{dV = -d\rho \frac{M}{\rho^2}}$ : second law, for astrophysicists
\end{list}

\subsection{Review of the Adiabatic Process}

\begin{align}
\epsilon = E/\text{unit volume}, NR: P &= \frac{2}{3} \epsilon\\
R: P &= \frac{1}{3} \epsilon\\
U = \frac{\epsilon}{\rho} = \frac{P}{\rho} = \phi U~,\text{ where $\phi$ is either $1/3$ or $2/3$}\\
dU = \frac{P}{\rho^2}d\rho = \phi U\frac{d\rho}{\rho}\\
\frac{dU}{U} = \phi \frac{d\rho}{\rho}\\
U \pt \rho^\phi~,\text{ for an adiabatic process}\\
P \pt \rho U \pt \rho ^{\phi+1} \pt \rho^{\gamma}~, \phi + 1 \text{ is the adiabatic index}\\
\end{align}

\begin{align}
\text{For a NR gas: } \phi = \frac{2}{3}, \gamma = \frac{5}{3}~,P\pt \rho^{5/3}, T \pt \rho^{2/3}\text{ for an adiabatic process}\\
\text{For a R gas: } \phi = \frac{1}{3}, \gamma = \frac{4}{3}~,P\pt \rho^{4/3}, T \pt \rho^{1/3}\text{ for an adiabatic process}
\end{align}

\subsection{What is the Entropy of an Ideal Gas?}

\begin{align}
TdS &= dU - \frac{P}{\rho^2}d\rho\\
\frac{TdS}{U} &= \frac{dU}{U} - (\gamma - 1)\frac{U\frac{d\rho}{\rho}}{U}\\
U &=\frac{P}{\rho}\frac{1}{\gamma -1} \frac{kT}{m}\\
\frac{m(\gamma -1)}{k}dS &= \frac{dU}{U} - (\gamma -1)\frac{d\rho}{\rho}\\
\frac{m(\gamma - 1)}{k}s &= \ln U - (\gamma - 1)\ln \rho + c\\
\Aboxed{s &= \frac{k}{m}\frac{1}{\gamma - 1} \ln \lp \frac{U}{\rho^{\gamma - 1}} \rp  +c}\\	
s &= \frac{k}{m} \frac{1}{\gamma -1} \ln \lp \frac{P}{\rho^\gamma} \rp + c
\end{align}

For an adiabatic process, $s=0$.

\begin{align}
\frac{Tds}{dt} &= \frac{dU}{dt} - \frac{P}{\rho^2}\frac{d\rho}{dt}\\
&= E_{fusion} - \frac{1}{\rho}(\bar{\nabla} \cdot \bar{F})
\end{align}

Say a blob is gaining/losing heat. $E_{fusion}$ is the heating per mass per time and $\bar{F}$ is the flux of $E$. In general:

\begin{align}
\text{total cooling} &= \int \bar{F}\cdot d\bar{A}\\
&= \int \bar{\nabla} \cdot \bar{F} d\bar{V}\\
\text{cooling per unit $V$} &= \bar{\nabla} \cdot \bar{F}\\
\text{cooling per unit mass} &= \frac{1}{\rho}(\bar{\nabla} \cdot F)
\end{align}

If a blob moves up a distance $dr$, given $T(r), P(r),$ and $\rho(r)$, is the fluid buoyantly stable? i.e. $\rho_{blob} \gtrless \rho_*$? We'll be making 2 assumptions which we will then confirm \textit{post-facto}. 
\begin{list}{}{}
\item Motion is adiabatic $\leftarrow$ valid if the time scale to move ($\sim$ 1 month) is sufficiently smaller than the time to exchange heat with the surroundings ($\sim10^7$ years)
\item $P_{blob} = P_*$ at all times; in pressure equilibrium with surroundings
\end{list}

The time scale to establish HE: $\sim\frac{1}{\sqrt{G\rho}} \sim 1 $ hr $\ll$ time to move $dr$, which is about a month. If it's adiabatic, $s_{blob} = s \ne s_*$ in general, where $s_{blob}$ is the blob at the new position, $s$ is the initial entropy, and $s_*$ uis the background entropy of the star at the new position.\\

\begin{center}
\begin{tabular}{c|c}
\hline
$\dfrac{ds}{dr} < 0$ & $\dfrac{ds}{dr} > 0$\\ \hline
$s>s_*$ & $s < s_*$\\ \hline
$s_{blob} > s_*$ & $s_{blob}< s_*$\\ \hline
$P_{blob} = P_*$ & $P_{blob} = P_*$\\ \hline
$\rho_{blob} < \rho_*$ & $\rho_{blob} > \rho_*$ \\ \hline
buoyancy unstable, rises & sinks back down (stable)\\
\hline
\end{tabular}
\end{center}

\chapter{More Convection}

\begin{center}
\textbf{\begin{huge} September 15, 2011\end{huge}}
\end{center}

\section{Models}
Blob's motion is adiabatic: $s_{blob} = s \ne s_*$. The blob is in pressure and pressure equilibrium with it's surroundings. The timescale for the blob to move is $\ll t_{dynamic}$ or $t_{sound~crossing}$. 

\begin{align}
s \pt \ln \lp \frac{P}{\rho^\gamma} \rp : \frac{ds}{dr}<0~, s_{blob} = s > s_*~,\rho_{blob} < \rho_*
\end{align}

At the new position, $a = g \frac{\Delta \rho}{ \rho} = |\bar{g} | \lp \frac{\rho_* - \rho_{blob}}{\rho_{blob}} \rp$. If $\rho_{blob} > \rho_*$, a is negative and it goes back down. If $\rho_{blob} < \rho_*$, then a = pos.

\subsection{Background Stellar Model}

\begin{align}
\boxed{\rho_* = \rho + \frac{d \rho}{dr} \delta r}~,\text{ where $ \frac{d \rho}{dr}$ is the $\rho$ gradient of the stellar model}
\end{align}

This, however, is quite boring. 

\begin{align}
P_* &= \rho + \frac{d\rho}{dr}\delta r\\
&= P_{blob} = P _ \delta P\\
\delta P &= \frac{dP}{dr}\delta r
\end{align}

\subsection{Blob Model}

\begin{align}
\rho_{blob} &= \rho + \lp \frac{\delta \rho}{\delta P} \rp_S \delta P\\
P &\pt \rho^\gamma\\
\lp \frac{\delta P}{\delta \rho} \rp_s &= \gamma \frac{P}{\rho}\\
\lp \frac{\delta \rho}{\delta P}\rp_s = \frac{\rho}{\gamma P}\\
\rho_{blob} &= \rho + \frac{\rho}{\gamma}\frac{\delta P}{P}\\
\Aboxed{\rho_{blob} &= \rho + \frac{\rho}{\gamma} \frac{d \ln P}{dr} \delta r}
\end{align}

\begin{align}
a &= g  \lp \frac{\rho_* - \rho_{blob}}{\rho_{blob}} \rp\\
&= g \lp \frac{\rho + \frac{d\rho}{dr} \delta r}{\rho + \frac{\rho}{\gamma} \frac{d \ln P}{dr}\delta r} - 1 \rp~,\text{ assuming that $\delta r$ is really small}
\end{align}

\begin{align}
a \equiv -N^2\delta r\\
N &= -g \lp \frac{d \ln \rho}{dr} - \frac{1}{\gamma} \frac{d \ln P}{dr} \rp\\
&= \frac{\gamma -1}{\gamma} \frac{m}{k}g \frac{ds}{dr}
\end{align}

$[N] = \frac{1}{s}$, the Brunt-V\"ais\"al\"a Frequency. Whether or not convection setes in or not is only dependant on $\frac{ds}{dr}$. 

\begin{align}
a = \frac{d^2}{dt^2}\delta r = -N^2 \delta r\\
\boxed{\frac{d^2}{dt^2}\delta r + N^2 \delta r = 0}~,\text{ Equation of Motion}
\end{align}

If $N^2 > 0$, then $\delta r \pt e^{iNt} \pt \cos(Nt) } + i\sin(Nt)$. This is a stable oscillatory solution. If you push it a little bit, it'll oscillate. If $N^2 > 0, \frac{ds}{dr} < 0$, which is an exponentially growing solution. $\delta r \pt e^{|N|t}$ occurs on a time scale $\frac{1}{N}$ and is dubbed "convection". Say for example we know $dP, d\rho, ds$, then we know if it's spontaneously boiling.\\

\subsection{Will Convection Set In?}

Convection sets in if:

\begin{align}
\frac{ds}{dr} < 0\\
N^2 < 0\\
s &\pt \ln \lp \frac{P}{\rho^\gamma} \rp\\
 &\pt \ln \lp \frac{T^\gamma}{P^{\gamma -1}} \rp\\
P = \frac{\rho kT}{\mu m_p}\\
\rho \pt \frac{P}{T}~.
\end{align}

\begin{align}
\gamma \frac{d \ln T}{dr} &< ( \gamma - 1) \frac{d \ln P}{dr}\\
\frac{d \ln T}{dr} &< \frac{\gamma -1}{\gamma	}\frac{d \ln P}{dr}\\
\Aboxed{\Bigl\lvert \frac{d \ln T}{dr} \Bigl\lvert &> \frac{\gamma -1}{\gamma} \Bigl\lvert \frac{d \ln P}{dr} \Bigl\lvert} \\
\text{or}\\
\Bigl\lvert \frac{d \ln T}{d \ln P	} \Bigl\lvert &> \frac{\gamma - 1}{\gamma} = \frac{2}{5}~,\gamma = \frac{5}{3}
\end{align}

We can calculate $\frac{dT}{dr}$ if the photons carry the energy.

\begin{align}
\Bigl\lvert \frac{d \ln T}{dr} \Bigl\lvert &> \frac{\gamma -1}{\gamma} \Bigl\lvert \frac{d \ln P}{dr} \Bigl\lvert\\
\text{Combine HE \& RD: } \frac{dP_{rad}}{dP} &= \frac{L_r \kappa}{4 \pi GM_r c}\\
\frac{P}{P_{rad}}\frac{dP_{rad}}{dP} &= \frac{L_r \kappa}{4 pi GM_r c}\frac{P}{P_{rad}}~,\frac{P}{dP} = \frac{1}{d \ln P}\\
\Aboxed{\frac{d \ln P_{rad}}{d \ln P} &=4 \frac{d \ln T}{d \ln P}}
\end{align}

Quick note about the $d \ln T$ stuff: take for example $P\pt T_{rad}^4$. $\ln P_{rad} \pt 4 \ln T +$ constant. $\frac{d}{d \ln T} \ln P_{rad} = \frac{d}{d \ln T}4 \ln T +$ constant$= 4$.
If photons carry the energy, can directly calculate: 

\begin{align}
\boxed{\frac{d \ln T}{d \ln P} = \frac{1}{4} \frac{P}{P_{rad}} \frac{L_r \kappa}{4 \pi GM_rc}}
\end{align}

Another way to think about it is that if 

\begin{align}
\frac{d \ln T}{d \ln P} > \frac{\gamma -1}{\gamma}~,
\end{align}

then convection sets in.

\section{Our $\odot$}

For our sun, convection occurs in a radius abour $r \gtrsim 0.7 R_\odot$. At the enter of the sun, $T\sim 10^$ K. At larger $r$, $T \downarrow$ and $\kappa \uparrow$. 

\begin{align}
\frac{d \lnT}{d \ln P} > \frac{2}{5}~,\text{ at higher $r$, convection is determined by $\kappa$}
\end{align}

In the region between $0 < r \lesssim 0.7R_\odot$, the sun is relatively much hotter than its other parts and thus photons can carry the energy out. The $l$ per photons is relatively large and streaming out isn't a problem. As you extend outwards, however, $T \downarrow, \kappa \uparrow$, resulting in a higher $N$ (where in this case $N$ is the number of RW steps). Photons now have trouble moving the energy and convection takes over. \\

$M < M_\odot, T < T_\odot, \kappa \uparrow$, and more convection. For stars with $M \lesssim \frac{1}{3}M_\odot$, they are fully convective. $M > M_\odot, T > T_\odot, \kappa \downarrow$, the surface convection zone goes away, BUT the core now becomes convective. What actually happens when convection moves energy?

\section{Energy Transport by Convection}

Let's say there are two blobs, one with a high $\rho$ and low $T$ and another with a low $\rho$ and a high $T$. The blobs with lhgh $\rho$ will sink and the blob with low $\rho$ will rise. This makes sense, hot stuff rises, cold stuff sinks. 

\begin{align}
v_c = \text{ typical convective velocity}\\
F_c &\sim \frac{1}{2}\rho v_c^2 \cdot v_c \\
\Aboxed{&\sim \frac{1}{2} \rho v_c^3}\\
\text{OR}\\
\Aboxed{F_c &\sim \rho \Delta E \cdot v_c}
\end{align}

Since $d\rho$ is so small, able to use either. Let's determine the v_c through the work done by buoyancy. When a blob rises from one position to another, it moves from a region of high $P$ to a region of lower $P$. The blob shared energy with it's surroundings and must enlarge to conserve mass. 

\begin{align}
l \equiv \alpha H~,\text{ H is the scale height},H = \lp \frac{d \ln P}{dr} \rp^{-1}
\end{align}

How much energy goes the blob gain?

\begin{align}
a \equiv |N|^2\delta r\\
E_{gained/mass} = \frac{1}{2}\rho v_c^2\\
v_c^2 &= |N|^2 \delta r\\
v_c^2 &= |N|^2l^2\\
&= \alpha^2 H^2 |N|^2~,|N|^2 = g\frac{1}{c_p} \frac{ds}{dr}
\end{align}

%%%%%%%%%%%%%%%%%%%%%%%%%%%

\chapter{Finishing Convection}

\begin{center}
\textbf{\begin{huge} September 20, 2011\end{huge}}
\end{center}

\section{Convection Continued}

\begin{align}
a = |N|^2 \delta r\\
|N|^2 &= \frac{g}{c_p} \Bigl\lvert \frac{ds}{dr} \Bigl\lvert\\
&= \frac{g}{H} \Bigl\lvert \frac{H}{c_p} \frac{ds}{dr} \Bigl\lvert\\
v_c^2 = a \delta r = |N|^2 \delta r^2\\
\delta r \equiv \alpha H\\
v_c = \alpha c_s \Bigl\lvert \frac{H}{c_p} \frac{ds}{dr} \Bigl\lvert^{1/2}\\
F = \frac{1}{2}\rho v_c^3 = \frac{1}{2} \rho \alpha ^3 c_s^3 \Bigl\lvert \frac{H}{c_p} \frac{ds}{dr} \Bigl\lvert ^{3/2}
\end{align}

We wanted to find the $F_r = -\frac{4}{3} \frac{caT^3}{\kappa \rho}\frac{dT}{dr}$ equivalent for convection. $F = \frac{1}{2}\rho v_c^3 = \frac{1}{2} \rho \alpha ^3 c_s^3 \Bigl\lvert \frac{H}{c_p} \frac{ds}{dr} \Bigl\lvert ^{3/2}$ gives the $v_c$ and $\frac{ds}{dr}$ given the flux. \\

$\Bigl\lvert \frac{H}{c_p} \frac{ds}{dr} \Bigl\lvert \sim 10^{-6}$, $s \sim c_p$, so $\frac{\Delta s }{s} \sim 10^{-6}$ on a length scale $~H$. Ergo, $s$ = constant in the convection zone. This replaces $F_r = -\frac{4}{3} \frac{caT^3}{\kappa \rho}\frac{dT}{dr}$. Let's assume $P \pt \rho^\gamma$ \& $\frac{dP}{dr} = -\rho \frac{GM_r}{r^2}$. 

\begin{align}
\frac{dM_r}{dr} = 4 \pi r^2 \rho\\
\frac{d}{dr} \lp \frac{r^2}{\rho} \frac{dP}{dr} = -\rho GM_r \rp\\
\frac{d}{dr} \lp \frac{r^2}{\rho	} \frac{dP}{dr} \rp = -4\pi r^2 G\rho
\end{align}

But... if $P = K \rho^\gamma$, $\frac{dP}{dr} = \gamma K \rho^{\gamma-1} \frac{d\rho}{dr}$! These kinds of models are called:

\subsection{Polytropic Models}

\begin{align}
P &= K\rho^\gamma\\
&= K \rho^{1 + 1/n}, \gamma = 1 + \frac{1}{n}, \text{ where n is the polytropic index}
\end{align}

\begin{align}
\theta = \lp \frac{\rho}{\rho_c} \rp ^{1/n}~,\rho_c = \rho(r=0)\\
\zeta = \frac{r}{a}~, a = \sqrt{\frac{(n+1) K \rho_c ^{\frac{1}{n} -1}}{4 \pi G}}~,[a] = \text{ cm}
\end{align}

\begin{align}
\boxed{\frac{1}{\zeta} \cdot \frac{d}{d\zeta} \lp  \zeta^2 \frac{d\theta}{d\zeta} \rp = -\theta^n}
\end{align}

Let's look at the properties of a fully convective star of low mass. Low mass $\rightarrow$ low $T$ $\rightarrow$ high $\kappa$. 

\subsection{$M_* < \frac{1}{3} M_\odot$ on MS}

For stars with photons carrying the energy out, $L \pt M^3$ if $\sigma = \sigma_T$ for fully convective stars, $L = 4 \pi R^2 F_c$, where $F_c = \rho v_c^3 \pt \Bigl\lvert \frac{ds}{dr} \Bigl\lvert^{3/2}$. Let's look at the surface where photons are carrying the energy out. 

\begin{align}
\text{s = constant}\\
P \pt \rho^{5/3} \pt T^{5/2}\\
\rho T \pt \rho^{5/3}, T \pt \rho^{2/3}\\
\frac{P_c}{P_{photons}} = \lp \frac{T_c}{T_{eff}} \rp ^{5/2},\text{ now use V.T. to relate $T_c$ and $M$ \& $R$.}
\end{align}
\begin{center}
{\Huge BUT}
\end{center}
We know that this is a $n=3/2$ polytrope so $kT_c = .54 \frac{GM\mu m_p}{R}$

\begin{align}
P_c = 0.77 \frac{GM^2}{R^4}\\
\frac{dP}{dr} = -\rho \frac{GM_r}{r^2}\\
\frac{P_c}{R} \sim \frac{M}{R^3}\frac{GM}{R^2}\\
P_c \sim \frac{GM^2}{R^4}
\end{align}

\begin{align}
\frac{1}{\kappa_{ph}\rho_{ph}} &= \frac{kT_{eff}}{mg} \\
&= \frac{kT_{eff}R^2}{GMm}\\
\frac{\rho_{ph}kT_{eff}}{m} &= \frac{g}{\kappa_{ph}}\\
&= P_{ph}\\
& = \frac{GM}{R^2\kappa_{ph}}
\end{align}

Now all that's left is $\kappa$. $\kappa_{ph}(\rho_{ph},T_{ph})~, \kappa_{ph} \sim 3 \times 10^{-31} \rho_{ph}^{1/2} T_{eff}^9$. Now we have $\frac{\rho_{ph} kT_{eff}}{m} = \frac{g}{\kappa_{ph}}$. In this example, $\kappa = \kappa_{H^-}$ and $T_{eff} \sim 3000 \sim 10^4$ K. Now we can solve:

\begin{align}
\frac{P_c}{P_{ph}}&= \lp \frac{T_c}{T_{eff}} \rp^{5/2}\\
T_{eff} &= T_c \lp \frac{P_{ph}}{P_c	} \rp ^{2/5}\rightarrow T_{eff}(M,R)~,
\end{align}

and from here you can solve for $T_c(M,R), P_c(M,R), P_{ph}(T_{eff},M,R)$.\\

We're going to assume that $s=$ constant and some other stuff about polytropes. $l \sim H \rightarrow P_{ph}(M,R,T_{eff})$ to get:

\begin{align}
\Aboxed{T_{eff} &\approx 3000 \text{ K }\lp \frac{M}{M_\odot} \rp^{1/7} \lp \frac{R}{R_\odot} \rp^{1/49}}\\
\Aboxed{L = 4 \pi R^2 \sigma T_{eff}^4 &\sim 0.1 L_\odot \lp \frac{M}{M_\odot} \rp^{4/7} \lp \frac{R}{R_\odot} \rp^{102/49}}~,\\
\text{ for a fully convective star; this is analogous to the $L \pt M^3$ for a radiative star}\\
T_{eff} &\approx 3000 \text{ K } \lp \frac{L}{L_\odot} \rp^{1/102} \lp \frac{R}{R_\odot} \rp^{7/51}
\end{align}

But the dependence on $M$ and $R$ are so low that fully convective stars share the same $T_{eff} \sim 3000-4000$ K and are pretty much located on the Hayashi Line.

%%%%%%%%%%%%%%
\chapter{Star Formation}

\begin{center}
\textbf{\begin{huge} September 22, 2011\end{huge}}
\end{center}

\section{Star Formation}

Gas in galaxies comes in multiple "phases". It's still a gas, just a broad range with particular characteristics. They have different $\rho$ \& $T$ with comparable $P$. Hot, low $\rho$ gas is mostly in the form of Hot ISM. Stars form from \textbf{cold molecular clouds}. What are the conditions for a cold molecular gas cloud to collapse? 

\begin{align}
|U| &\geq |K|\\
\frac{GM}{R^2} &\gtrsim \Bigl\lvert \frac{dP}{dr} \Bigl\lvert\\
\text{self gravity of cloud} &\gtrsim \frac{3}{2}NkT\\
&\approx \frac{M}{m_p}kT
\end{align}

If $M \gtrsim \frac{RkT}{Gm_p}$, then it will collapse. 

\begin{align}
\rho &\approx \frac{M}{R^3}\\
R &\sim \lp \frac{M}{\rho} \rp^{1/3}\\
M^{2/3} &\gtrsim \frac{kT}{Gm_p \rho^{1/3}}\\
\Aboxed{M_J &\geq \lp \frac{k}{Gm_p} \rp^{3/2} \frac{T^{3/2}}{\sqrt{\rho}}}~, \text{ Jeans Mass}\\
\frac{GM^2}{R} &\geq \frac{MkT}{m_p} \\
\frac{GM^2}{R} &\geq c_s\\
\text{if } \frac{1}{\sqrt{G\rho}} &\leq \frac{R}{c_s}~,\text{ then $t_{FF} < t_{sound}$, and it will collapse}
\end{align}

Stars are more prone to collapse if they have lower $T$ and higher $\rho$. Stars form from cold molecular clouds because they are the most unstable.

\begin{align}
M_J &\approx 50 M_\odot \frac{\lp \frac{T}{10K}\rp^{3/2}}{\lp \frac{\text{n}}{100 \text{ cm}^{-3}} \rp^{1/2}}\\
R_J &= \lp \frac{M_J}{\rho} \rp^{1/3}\\
&\approx 3 \text{ pc} \frac{(T/10K)^{1/2}}{(n/100\text{ cm}^{-3})^{1/2}}
\end{align}

If a star has $M>M_J$ and $R<R_J$, then it will collapse. The collapse time is $\sim \frac{1}{\sqrt{G\rho}} \sim 10\text{ Myr} \lp \frac{n}{100 \text{ cm}^{-3}} \rp^{-1/2}$. Why don't we have tons of $50M_\odot$ stars? In reality, most of them are roughly $0.3M_\odot$. 

\begin{align}
\rho a &= -\frac{dP}{dr}- \rho \frac{GM}{R^2}\\
&\sim \frac{P}{R} - \frac{GM^2}{R^5}\\
&\pt \frac{nT}{R}\\
&\pt \frac{MT}{R^4}
\end{align}

\section{Actual Collapse}

Initially, the gas cools rapidly and since photons easily escape cloud, $T \sim$ roughly constant, isothermal at around $10K$. $P \pt \frac{M}{R^4}$, \& gravity $\pt \frac{M^2}{R^5}$. As radius decreases, gravity dominates. There is clearly no halt to the collapse while $M=$ constant. What's happening to $M_J \pt \frac{T^{3/2}}{\sqrt{\rho}}$? So during this procss, $M_J \downarrow, \rho \uparrow$. Little regions within the cloud collapse on themselves, so what determines the mass of the small cloudlets? \\

Now the cloudlets are sufficiently dense so that energy can't escape. The cloud becomes adiabatic, $\kappa \uparrow, l \downarrow, T \uparrow$. The random walk time of the photons is now longer than $t_{ff}$; for adiabatic:

\begin{align}
P &\pt \rho^\gamma\\
\rho &\pt \rho^\gamma\\
T &\pt \rho^{\gamma-1} = \rho^{2/3}\\
T &\pt \frac{M^{2/3}}{R^2}
\end{align}

Now, plug in the above expression for $T$ into $P \pt \frac{MT}{R^4}$ to get $P \pt \frac{M^{5/3}}{R^6}$. Yay! No more runaway collapse. Now, $\frac{dP}{dr} \sim -g\rho$ and were back in HE. $M_J \uparrow, T\pt \rho^{2/3}, M_J \pt \rho^{1/2}$.\\

The cloud is now in HE, but $R \gg R_\odot$. There exists a luminosity of the cloud from simply losing $E$. This contraction will (eventually) lead to fusion.

\begin{align}
E_{TOT} = \frac{U}{2}\\
L = -\frac{dE}{dt} = -\frac{1}{2} \frac{dU}{dt}\\
U = -\frac{GM^2}{R}\\
L = \frac{GM^2}{2R^2} \Bigl\lvert \frac{dR}{dt} \Bigl\lvert
\end{align}

If the cloud is radiative, $L \pr M^3$. If the cloud is convective, $L \simeq 0.2 L_\odot \lp \frac{M}{M_\odot}\rp^{4/7} \lp \frac{R}{R_\odot}\rp^2$. 

\begin{align}
L_{conv} = \frac{1}{2}\frac{GM^2}{R^2} \Bigl\lvert \frac{dR}{dt} \Bigl\lvert\\
\Aboxed{\frac{R}{R_\odot} &\approx \lp \frac{M}{M_\odot}\rp^{1/2} \lp \frac{t}{2 \times 10^7 \text{ years}} \rp^{-1/3}}\\
R &\pt t^{-1/3}
\end{align}

Plug in the above expression for $\frac{R}{R_\odot}$ into $L \simeq 0.2 L_\odot \lp \frac{M}{M_\odot}\rp^{4/7} \lp \frac{R}{R_\odot}\rp^2$ to get:

\begin{align}
L&\approx 0.2 L_\odot \lp \frac{M}{M_\odot}\rp^{11/7} \lp \frac{t}{2 \times 10^7 \text{ years}}\rp^{-2/3}\\
L &\pt t^{-2/3}
\end{align}

As the cloud contracts, $L \downarrow, R \downarrow$. For a fully convective cloud, $L \uparrow$ if $R \uparrow$. They all have similar $T_{eff}$. $L \sim 4 \pi R^2 T_{eff}^4$, $L \pt R^2$, which is what we have.

%%%%%%%%%%%%%%%%%%%%%%%
\chapter{Thermonuclear Fusion}

\begin{center}
\textbf{\begin{huge} September 27, 2011\end{huge}}
\end{center}


\section{Thermonuclear Fusion}
Read Phillips 1.4, 1.5, 4.1, 4.2\\

Nuclei-\\
\begin{list}{*}{}
\item Z - proton \# 
\item A - mass \#
\item N - \# of neutrons (N= A-Z)
\item Proton mass = $m_p c^2 = 938.259$ MeV
\item Neutron mass = $m_n c^2 = 939.553$ MeV
\item 1 MeV 1.6 $\times 10^{-6}$ ergs
\item Isotope - same Z (Carbon 12, 14 are isotopes)
\item Isobar - same atomic \# (Carbon 14, Nitrogen 14)
\item size of nucleus $\sim$ 1.3 $A^{1/3}$ fm  ($10^{-15} m = 10^{-13} cm$)
\end{list}

Free n can $\beta$ decay
\begin{equation}
n \rightarrow p + e^- + \bar{\nu_e}~, \text{will decay in something like 900 s}
\end{equation}

Tells us nuclei have constant density... so we take the...

\begin{align}
\rho & = \frac{A m_p}{\frac{4 \pi}{3}r_n^3} \\
& = \frac{m_p}{\frac{4 \pi }{3}1.3 ~\text{fm}}\\
& = 10^{14} \text{g cm}^{-3}
\end{align}

Size set by strong force. Falls off quickly for $r > r_n$. \\

Some of the forces we'll take about are the lon-range forces. (Gravity, E\&M...) The particle transmitting the force has to be massless. IN E\&M, it's the photon. In gravity, it's the graviton. Its the fact that these particles are massless lets us hve this long range force. For TN, we have the strong force... but the particle which mediates the force has a mass. \\

Finite rest mass $\rightarrow$ short range force\\ 

\begin{eqnarray}
\Delta E \Delta t \sim \bar{h} \rightarrow \Delta t \sim \frac{\bar{h}}{E}\\
d \sim c \Delta t\\
E \sim \frac{\bar{h}c}{d} \sim \frac{ 197 ~\text{MeV fm}}{1 ~\text{fm}} \sim 200~ \text{MeV}
\end{eqnarray}

This particle turns out to be the pion. $m_\pi c^2 \sim 140$ MeV. \textbf{When are the nuclei stable? }

\subsection{Nuclear Stability}
Not every Z \& A are stable. There's a region of stability for low Z, elements that have Z = N are stable and at higher Z, N $\geq$ Z are stable. But why? Let's use shell nucleus (remember n \& p, like $e^-$) \textit{Pauli Exclusion Principle!} Use the shell mode to build out. \\

The main difference between the shell model for electron and the shell model for the nucleus is that... it's different. There are two energy levels, on for each proton and neutron, for example. n can $\beta$ decay, $n \rightarrow p + e^- + \bar{\nu_e}$. The neutron move to lower energy states and into protons. Sometimes you can also reach total lower energy if you turn a spare proton in its own energy level into a neutron to pair with on that's alone. $ p \rightarrow n + e^+ + \nu_e$. Basically, you want to put things in the lowest energy level. This process favors Z $\sim$ N. Now this breaks down... but why? \\
Once you get to massive nuclei, the Coulomb repulsion starts kicking in. The EM repulsion wants to fight back. More neutron means more strong force, which lets you hold together the protons which are repelling. If you want to build stable nuclei, you're going to need to add more neutron than protons. How much energy is holding nuclei together? (Binding energy) Held together by strong force. 

\begin{align}
E_\text{nuc} & = Zm_pc^2 + N m_n c^2 -E_b = m_\text{nucleus}c^2~,\\
\frac{E_b}{A} & = \text{binding energy per nucleon}\\
& \approx 8~\text{MeV (peaks at Fe 56 and secondly at Ni 62)}
\end{align}

For He$^4$, $\frac{E_b}{A} \approx 7$ MeV. 

\subsection{Fusion}
light + light $\ra$ heavier + energy\\
Happens for $A \leq 60$. If yout try to do lead + lead, don't get anything out. Less bound + less bound $\ra$ more bound + energy, to put it better. What sets the nuclear energy scales? \\

Nuclear energy scale:\\
Coulomb repulsion of proton:\\
\begin{align}
E & \sim \frac{(Ze)^2}{r}~,\\
& \sim \frac{(Ze)^2}{A^{1/3} ~ \text{fm}} \sim 1 ~ \text{MeV}
\end{align}

What's the Fermi energy of the nucleon?

\begin{align}
E_F & \sim \frac{3}{5} \frac{p_F^2}{2 m_p}~, p_F \sim \left( \frac{3 n^{1/3}}{8 \pi} \right) h\\
& = 25 ~\text{MeV}
\end{align}

Now that we've done nuclear physics, doing order of magnitude for fusion.

\subsection{Order of Magnitude Estimates for Fusion}

What's the dominant reaction (In the sun)? More or less, it's $4 \text{p} \ra ^4$He. The BB gives lots of H. The binding energy of Helium is about 28.3 MeV. How much energy can we get out of the sun if all of its H turns into He? 

\begin{align}
E_\odot = 28~\text{MeV} \left( \frac{m_\odot}{4m_p} \right) = 10^{52} ~\text{ergs}\\
t_{nuc} \sim \frac{E_\odot}{L_\odot} \sim \frac{10^{52}~\text{ergs}}{4 \times 10^{33} ~\text{ergs}} \sim 3 \times 10^8 ~ \text{s} \sim 10^{10} ~ \text{yr}
\end{align}
However,  only fuses H $\ra$ He in the central 10\% of the sun (by mass). \textbf{Why is fusion hard?}

\subsection{Why is Fusion Hard?}
The impediment is Coulomb repulsion. We have to exert some energy to push them together then let strong force take over and fuse them together. \\

Two protons are separated by a distance r.

\begin{align}
E = \frac{1}{2} \mu v^2 + \frac{e^2}{r}~, E \approx kT
\end{align}

When the proton are closest together, their $v=0$. How much energy do we need to get the $p$ nuclearly close together? 

\begin{align}
E \approx \frac{v^2}{r_c} \approx kT~, r_c \sim \frac{e^2}{kT}
\end{align} 

We want $r_c < r_n$. $\ra$ $kT \geq \frac{e^2}{r_n} \ra T \geq 10^{10}$K. But the central $T_\odot$ is only $10^7$ K! Did we do something wrong? Yeah! Why? We didn't use QM.\\
How do we estimate if QM is important? \\

deBroglie Wavelength:
\begin{align}
\lambda \sim \frac{h}{p} \sim 10^{-10} \left( \frac{T}{10^7 ~\text{K}} \right)^{-1/2}~ \text{cm}
\end{align}

If $\lambda \gg r_n$, then we cannot ignore QM. How high does our energy have to be to get over the barrier? What barrier? Oh, you don't have a graph. Lawls. \\

There's some finite probability that we can get through the potential hump where $E \ll \frac{e^2}{r_n}$ and reach $r_n$ \& feel strong force. Tunneling is important when:

\begin{align}
r_c \sim \lambda ~.
\end{align}

What is $r_c$? It's defined as $\frac{Z_1 Z_2 e^2}{r_c} \sim \text{kT} \sim \frac{p^2}{2m}$. 

\begin{align}
r_c & = \frac{Z_1 Z_2 e^2 (2m)}{p^2} \sim \lambda \sim \frac{h}{p}~, p = \sqrt{\text{kT} m_p}\\
\text{kT} & \sim \frac{4 Z_1^2 Z_2^2 e^4 m_p A}{h^2}\\
\text{T} & \sim 3 \times 10^7 Z_1^2 Z_2^2 ~\text{K}~, \text{which is more on order of magnitude of central T}
\end{align}

\section{Big Picture}
Took HE and found kT $=\frac{Gm \mu m_p}{3R}$. Took radiative diffusion $L = M^3$ and fusion: T $\sim~10^7$ K. KH contraction of $M_\odot$ of gas: $R \gg R_\odot$, $T \ll T_\odot$, R down, T up. $L_{\text{fusion}} = L_{\text{radiative diffusion}}$. 


%%%%%%%%%%%%%%%%%%

\chapter{Post Fusion}
 
\begin{center}
\textbf{\begin{huge} September 29, 2011\end{huge}}
\end{center}

\boxed{\text{Phillips}~ 1.4,1.5,4.1,4.2}\\
\section{Post-Fusion}

\begin{align}
\lambda \sim \frac{h}{p} \gg & r_n \approx 10^{-13} ~\text{cm}\\
& r_c ~,~\text{classical distance distance, of closest approach}
\end{align}
Fusion parameters will be QM, but won't have stuff to do with $\lambda$. $\sigma$ for nuclear reactions $\approx$ nuclear physics part (strong/weak interaction) $\cdot$ tunnelling through coulomb barrier. We're going to focus on the tunnelling which sets the physics for the central temps of stars. \\

Let's consider SE:

\begin{align}
\left(\frac{\hbar^2}{2m_r} \nabla^2 + V(r) \right) \Psi = E\Psi~,m_r = \text{reduced mass}\\
-\frac{\hbar^2}{2m_r}\frac{d^2}{dr^2} \Psi =E\Psi~, \Psi = e^{ikr}~, E=\frac{\hbar^2 k^2}{2m_r}~,k=\frac{\sqrt{E2m_r}}{\hbar}
\end{align}
Good ol review. $P = | \Psi|^2 = \text{constant}$. Now we have

\begin{align}
\frac{\hbar^2}{2m_r}\frac{d^2}{dr^2} \Psi & = (V-E)\Psi~,\text{and} (V-E)>0\\
\Psi & \propto e^{-kr}~, \frac{\hbar^2 k^2}{2m_r} = (V-E)
\end{align}

QM'ally, particle can't be somewhere where it's potential is less than the energy. Now, the probability of tunnelling is $|\Psi|^2 \sim e^{-2kl}$. Tunnelling is generic feature of wave theory not just QM. Sound waves tunnel, waves in the atmosphere tunnel.... WUT. \\
Now lets imagine a particle with energy $E = \frac{1}{2}m_r v^2$,$v = | \bar{v_1} - \bar{v_2} |$. Now...

\begin{align}
\left( -\frac{\hbar^2}{2m_r}\nabla^2 + \frac{Z_1 Z_2 e^2}{r} \right) \Psi = E \Psi\\
E&=V\\
& = \frac{Z_1 Z_2 e^2}{r_c}\\
r_c &= \frac{Z_1 Z_2 e^2}{E}
\end{align}
There's some finite prob that they can tunnel trough the potential one they reach $r_c$. We want to compute the probability! One small difference is that with the atom, particles can have angular momentum and we have to use spherical harmonics.

\begin{align}
\Psi = \frac{f(r)}{r} Y_{l,m}(\theta,\phi)
\end{align}

OMG.\\

\begin{align}
\left( -\frac{\hbar^2}{2m_r} \frac{d^2}{dr^2} + \frac{l(l+1)\hbar^2}{2m_r r^2} + \frac{e^2 Z_1 Z_2}{r} \right) \Psi = E \Psi
\end{align}

Particles with high angular momentum don't fuse because any small difference in path and they'll fly off. Only possible fusion happens when $l=0$, so we can cross out the $\frac{l(l+1)\hbar^2}{2m_r r^2} $ term. Now... 

\begin{align}
\left( -\frac{\hbar^2}{2m_r} \frac{d^2}{dr^2} +  \frac{e^2 Z_1 Z_2}{r} \right) f = Ef
\end{align}

\textbf{What's the probability of reaching $r_n$ if they start at the classical turning point ($r_c$)?} $P = | f(r_n)|^2$. 

Now, 

\begin{align}
\frac{d^2 f(r)}{dr^2} + g(r)f(r) = 0~,g(r) = \frac{2m_r}{\hbar^2} \left( E - \frac{e^2 Z_1 Z_2}{r} \right)
\end{align}

We're interested in situations where the $E$ is less than the potential, so $g(r) < 0$. This pops up in  lot of places, apparently. If $g(r)$ is a constant, we can solve it. 

\begin{align}
f \sim e^{\pm i \sqrt{g} r}
\end{align}

This solution isn't vaid if $g(r)$ isn't a constant. For the case of interest, $g$ is \textit{almost} constant. It's slowly varying, in reality. It's a function of position for which there is an analytic solution to the above equation. The analytic equation is known as the WKB solution.\\

It's plausible that the solution is of the form $f \sim e^{i \phi(r)}$ if we think$g$ doesn't change much over time. If $g$ = constant, $\phi(r) = \sqrt{g} r$. 

\begin{align}
f' & = i \phi'(r)e^{i \phi(r)} = i \phi'(r)f\\
f'' & =i\phi''f + i \phi'f' = i \phi''f - (\phi')^2f
\end{align}

\begin{align}
\frac{d^2 f(r)}{dr^2} + g(r)f(r) & = 0\\
i\phi'' - (\phi')^2 + g&=0
\end{align}
Assume $\phi''$ is small, and by small we mean $\phi'' \ll g$. 

\begin{align}
(\phi')^2 &= g(r)\\
\phi' & = \sqrt{g(r)}\\
\phi(r) & = \int^r \sqrt{g(x)}dx\\
f & \sim e^{i\phi(r)} = e^{ \pm i\int^r \sqrt{g}dx}
\end{align}

We can check whether our assumption that $\phi'' \ll g$, $\phi'' = \frac{1}{2}g^{-1/2}g'$, and WKB solution is valid if $\frac{g'}{\sqrt{g}} \ll g$. This is what we mean by a "slowly varying" potential. Lets think about this physically. 

\begin{align}
g' &\sim \frac{g}{L}~,L = \text{length over which potential varies}\\
\frac{1}{L\sqrt{g}} &\ll 1\\
\frac{1}{\sqrt{g}} & \ll L\\
\phi &= \int \sqrt{g}dx\\
\phi &= \int \frac{dx}{\lambda}~,\text{where $\lambda$ is the wavelength to our solution on order $\frac{1}{\sqrt{g}}$}
\end{align}

Our WKB solution is okay if $\frac{1}{\sqrt{g}} \ll L$, $\lambda \ll L$. In our case, $\lambda$ is the deBroglie wavelength. 

\begin{align}
g = \frac{2m_r}{\hbar^2} \left( E - \frac{e^2 Z_1 Z)_2}{r}\right)
\end{align}


\subsection{Tunnelling}

\begin{align}
f(r_n) = e^{i \int_{r_n}^{r_c} \sqrt{g} dr} = e^{-\int_{r_n}^{r_c} \sqrt{|g|}dr}\\
P = e^{-I}~, I &= 2\int_{r_n}^{r_c} \sqrt{|g|}dr\\
I &= \frac{2\sqrt{2m_rE}}{\hbar} \int_{r_n}^{r_c} \left(  \frac{e^2 Z_1 Z_2}{r} -E \right)^{1/2} dr\\
& = \frac{2\sqrt{2m_rE}}{\hbar}    \int_{r_n}^{r_c} \left( \frac{r_c}{r} -1 \right)^{1/2}dr\\
x = \frac{r}{r_c}\\
I & = r_c \int_{r_n/r_c}^{x=r/r_rc} \left(\frac{1}{x}-1\right)^{1/2}dx\\
\int_0^1 \left(\frac{1}{x}-1\right)^{1/2}dx = \frac{\pi}{2}
\end{align}

Tunnelling is independent of where nuclear reaction becomes important. Tunnelling dominates at classical point. Once you get trough the turning point, it doesn't matter how far you have to go. 

\begin{align}
I = \pi \sqrt{ \frac{2m_r e^4 Z_1^2 Z_2^2}{\hbar^2 E} } = \left(\frac{E_g}{E} \right)^{1/2}~, E_g &= \frac{2\pi^2 m_r e^4 Z_1^2 Z_2^2}{\hbar^2}\\
E &\sim E_g~, I \sim 1~,\text{Prob of tunnelling}~ \sim 1\\
E& \ll E_g~, I \gg 1, P \ll 1\\
E_g = 1 ~\text{MeV} \frac{M_r}{m_p}Z_1^2 Z_2^2\\
P \approx e^{-(E_g/E)^{1/2}}
\end{align}
If $E$ is too low, no significant tunnelling and no significant fusion.

At center of sun....

\begin{align}
T_{center} \sim 10^7 \text{K} \sim 1 \text{KeV}\\
\frac{3}{2}kT &= 2\text{keV}\\
& \ll E_g \sim 500 \text{keV}\\
P \sim 10^{-7}~,\text{damn, that's low.}
\end{align} 

Let's imagine particles with 10 times the thermal energy. $E = 10 E_{th} = 20\text{keV}$, then $P \sim 10^{-2}$. This tells us that clearly particles whih are more energetic that average are \textbf{MUCH} more likely to tunnel and thus undergo fusion. So when do we treat things QM'ically? If deBroglie wavelength is large. Recap, $\lambda = \frac{h}{p} \sim \frac{h}{\sqrt{2Em_r}}$. As $E \uparrow$, $\lambda \downarrow$. Higher E has a smaller classical turning point. Then, $r_c \downarrow$ as $E \uparrow$. Yes, in absolute cm that the $r_c$ goes down, but relatively, it's easier to tunnel at higher $E$. 

\begin{align}
I &= (E_g/E)^{1/2}\\
& = \frac{\pi \sqrt{2m_rE}r_c}{\hbar} \sim \frac{r_c}{\lambda}
\end{align}

At high Z, $E_g \uparrow \rightarrow T \uparrow$. H is easiest to fuse at earliest stages.

%%%%%%%%%%%%%%%%%%%%%%%

\chapter{Finishing Fusion}

\begin{center}
\textbf{\begin{huge} October 6, 2011\end{huge}}
\end{center}


\boxed{ \text{Readings: Phillips}~ 5.1, 2.1,2.2,5.4}
\section{FUSION}
\subsection{PP Chain}
We're going into more detail on how to go from $4p \rightarrow ^4 He +$ energy, where energy is the KE of particles with a $l \ll R$. There are 2 key ways for the above reaction to occur... one is $p \rightarrow n$, such as beta decay ($n \rightarrow p + e^- + \overline{\nu_e}$). Anything that goes from $ p$ to $n$ has some relation to the weak interaction force. We want the ooposite too... where $p + p \rightarrow ^2H + e^+ + \nu_e$. where $^2H$ is Deuterium. 
\begin{align}
S \approx 3.78 \times 10^{-23}~\text{keV barn}
\end{align}

From now on, the "$\rightarrow$" will be the latter reaction. The pp chain consists of:

\begin{align}
p + p &\rightarrow ^2H + e^+ + \nu_e\\
^2H + p &\rightarrow ^3He + \gamma\\
^3He + ^3He &\rightarrow ^4He + 2p
\end{align}

Important to note here that the first two reactions must happen twice per 1 of the last one. Since there are no neutrinos, the last two reactions are based on the strong force. This entire cycle produces about 26.7 MeV, but a few \% comes out in the form of neutrinos. \\
Let's find the ergs/s/gram.

\begin{align}
\epsilon \propto \rho T^{-2/3}e^{-3(E_g/4kT)^{1/3}}
\end{align}

Let's recall the $l$ of a reaction, which is $l=\frac{1}{n\sigma}$. $T \approx \frac{l}{v} = \frac{1}{n \sigma v}$. Therefore, the pp step has a $\sigma$ which is must smallerthan the other steps. It means almost all the time, most of the reactions in the sun are waiting around for the initial step to happen so that you get $^H$. The latter steps happen almost instantaneously! The time for the entire cycle (pp chain) is set by the $p + p \rightarrow ^2H + e^+ + \nu_e$. This means we can write:

\begin{align}
\epsilon (\text{energy of entire chain}) = r_{12} Q / \rho ~,\text{where}~ r_{12}= n_1n_2<\sigma v>\\
\\
p + p \rightarrow ... E_g = 1/2 ~\text{MeV}\\
3 (E_g/4kT) = 15.7T_7^{-1/3}\\
\epsilon_{pp}  \propto \rho T^{-2/3} e^{-15.7 T_7^{-1/3}}\\
\epsilon_{pp}  \approx 5 \times 10^5 \frac{\rho X^2}{T_7^{2/3}}e^{-15.7 T_7^{-1/3}}~\text{ergs/s/g}\\
\epsilon \propto \rho T^\beta~,\beta = -2/3 + 5.2T_7^{-1/3}
\end{align}
and in stars like our sun where $T \sim 10^7$ K, $\beta = 4.5$ and $\epsilon \propto T^{4.5}$. Going to use this to estimate the central T of the sun. 

\begin{align}
L &= \int_0^M \epsilon dM_r \approx 0.1 \epsilon_c(r=0)M
\end{align}

use .1 because only 10\% of the mass of the sun contributes to L

\begin{align}
L &= 10^6 \frac{M}{T_7^{2/3}}e^{-15.7T_7^{-1/3}}
\end{align}

Solving for T, we get $T_c \approx 1.5 \times 10^7$K. Our $T_c$ is dependent only on the log of the uncertainty of how much of the sun is fusing. That's one way of going from 4 protons into a He...

\subsection{CNO Cycle}

\begin{align}
^{12}C + p & \rightarrow ^{13}N + \gamma \\
^{13} N & \rightarrow ^{13}C + e^+ + \nu_e\\
^{13}C + p  & \rightarrow ^{14}N + \gamma\\
^{14}N + p & \rightarrow ^{15}O + \gamma\\
^{15}O & \rightarrow ^{15}N + e^+ + \nu_e\\
^{15}N + p & \rightarrow ^{12}C + ^4He
\end{align}

Once again, reactions with $\nu_e$ are weak interactions and everything else are strong reactions. In this cycle, line 18 is the slowest step. The first beta decay has a $\lambda$ of 870 sec and the last beta decay has a $\lambda$ of 180 secs. These steps are weak, but still faster than the strong interaction steps. Let's look at how line 18 determines the rate of the CNO cycle. Both the pp chain and the CNO cycle are relevant for generating $ 4p \rightarrow ^4He$. 

\begin{align}
^{14}N + p \rightarrow  ^{15}O + \gamma \\
E_g = 45.7 ~\text{MeV}\\
3 \left( \frac{E_g}{4kT}\right)^{1/3} \approx \frac{70.7}{T_7^{-1/3}}\\
\epsilon_{CNO} & \propto \rho T_{-2/3} S_{CNO} e^{-70.6 T_7^{-1/3}}\\
\epsilon_{pp}& \propto \rho T_{-2/3} S_{pp}e^{-15.7 T_7^{-1/3}}
\end{align}

$S_{CNO} \sim 10^{24} S_{pp}$, so even if the slowest step in the CNO cycle is a strong process, not a weak process, the extremes cancel each other out. 

\begin{align}
\epsilon &= r_{\text{slowest chain}} Q/\rho\\
\epsilon_{CNO} &= 4.4 \times 10^{27} \frac{\rho XZ}{T_7^{2/3}} e^{-70.7 T_7 ^{-1/3}}~,\text{where}~ Z = ~\text{mass fraction of heavy elements}
\end{align}

Right at the beginning, stars couldn't use the CNO cycle, but later they are. \\

\begin{align}
\text{sun:} &\epsilon_{pp} \propto ^{4.5}\\
\text{CNO:} &\beta = \frac{\delta \ln \epsilon}{\delta \ln T} = -2/3 + 23.6 T_7^{-1/3}~, \epsilon \propto T^\beta\\
& \approx 23~ @~ 10^7 K\\
& \approx 17~@~ 2.7 \times 10^7K\\
& \propto T^{20}
\end{align}

At high T, CNO dominates. At low T, pp dominates. Stars a little more massive that the sun are dominated by CNO, whereas stars a little less massive then than the sun are dominated by the CNO cycle. \\

In the case of the sun, we have direct evidence to see the effects of fusion. How? Detection of neutrinos! Fusion is dominated in our sun by the pp chain and not the CNO cycle. We also see from neutrinos the approximate central T of the sun. \\

The neutrino matter cross section is dependent on the energy. At the center of the sun, the neutrino $l$ is about $10^9~R_\odot$. They give us the best probe about the center of the sun. \\

Ray Davis
\begin{align}
^{37}Cl + \nu_e \rightarrow ^{37}Ar + e^-
\end{align}

%%%%%%%%%%%%%%%%%

\chapter{Main Sequence}

\begin{center}
\textbf{\begin{huge} October 11, 2011\end{huge}}
\end{center}

$\boxed{5.1,2.1,2.2,5.4}$

\section{Main Sequence}

Know how to calculate $\epsilon(\rho,T)$ in units of ergs/s/g. Quickly review the major points of the MS:

\begin{align}
\boxed{\frac{dP}{dr}=-\rho \frac{GM_r}{r^2}=-\rho g}\\
P &= P_{gas} + P_{rad} (+ P_{degen})\\
&= \frac{\rho kT}{\mu m_p} + \frac{1}{3}aT^4\\
\boxed{\frac{dM_r}{dr} = 4\pi r^2 \rho}\\
E_{tot} = U/2 = -K, \text{ for NR}\\
E_{tot} \approx 0 , K = -U\text{ for R}
\end{align}

\subsection{Energy Transport (Radiation)}

\begin{align}
F_r = \frac{L_r}{4 \pi r^2} = -\frac{4}{3} \frac{caT^3}{\kappa \rho} \frac{dT}{dR}~,\kappa = \text{ opacity}\\
l = \frac{1}{\kappa \rho} = \frac{1}{n \sigma},\kappa = \frac{\sigma}{m}~,m = \text{ average mass of a particle}\\
\kappa_T,\kappa_{ff},\kappa_{bound~free},\kappa_{H^-},...
\end{align}

\subsection{Energy Transport (Buoyancy)}

Convection sets in if $\frac{ds}{dr} < 0$. Exponentially driven instability is driven by buoyancy of matter. Whether or not its convecting is dependent on the entropy gradient. Another way to put it is:

\begin{align}
\frac{d \ln T}{d \ln P} > \frac{\gamma -1}{\gamma}
\end{align}

We can get away with rough estimates using mixing length to find the work done by the buoyancy force. \\
For convective flux:

\begin{align}
F &= \frac{1}{2}\rho v_c^3\\
&= \frac{1}{2} c_s^3 \Bigl\lvert \frac{H}{c_p} \frac{ds}{dr} \Bigl\lvert ^{3/2}~,
\end{align}

when convection is present, and

\begin{align}
\boxed{\frac{1}{T}\frac{dT}{dr} = \frac{\gamma -1}{\gamma} \frac{d\rho}{dr}}
\end{align}

This is useful for fully convective objects because then $P \pt \rho^\gamma \pt \rho^{5/3}$. This is an example of the $n=3$ polytrope. We can therefore computer $T(r)$ and $\rho(r)$ (relatively) easy.

\subsection{Energy Generation in Stars}


Gravity: KH contraction, at a minimum. 

\begin{align}
\boxed{L = -\frac{1}{2}\frac{dU}{dt}\approx -\frac{GM^2}{R^2} \Bigl\lvert \frac{dR}{dt} \Bigl\lvert}
\end{align}

For the sun, $t_{KH} \approx 30$ million years. This contraction drives $T_c$ up and eventually fusion sets in. $\epsilon(\rho,T,\text{composition})$ is the energy generation by fusion. Fusion is a collisional process and you need both high densities and temperatures for two particles to get close enough to tunnel through their Coulomb Barriers. 

\begin{align}
L \approx \epsilon dM_r = \int_0^R 4 \pi r^2 \rho \epsilon dr
\end{align}

H fusion in the sun lasts about 10$^{10}$ years, which is about 3 orders higher than KH contraction. i.e. fusion is much more important than KH for luminosity. The variables we care about are: $P, \rho, T , L_r , M_r$ and the equations are: HE, Equation of State, $dM_r = 4\pi r^2 \rho$, energy transport, energy generation. While these equations are good, we need boundary conditions/initial conditions. If you specify the mass and initial composition of a star, that determines \textit{everything} $(T_c,T_{eff},\rho,R,L,T(r),P(r),...)$. For KH contraction, we could calculate $R(M, t)$ and $L(M, t)$.

%%%%%%%%%%%%%%%%%%
\chapter{Understanding Stellar Evolution}

\begin{center}
\textbf{\begin{huge} October 13, 2011\end{huge}}
\end{center}

The balance we're interested in is:

\begin{align}
L_{fusion} &\approx L_{rad/conv}\\
	& + 	\text{HE}\rightarrow\text{main sequence}
\end{align}

For stars of $M \leq M_\odot$, supported by $P_{gas}$, pp chain fusion, and $\kappa \sim \kappa_{ff}$. And if $\gamma$ carry energy out $L_{rad} \propto \frac{ M^{5.5} }{\sqrt{R}}$. Estimating $L_{fusion} = \epsilon_c M$. In the case of pp fusion, $\epsilon_{pp} \propto \rho T_c^{4/5}$ where $kT_c \sim \frac{GM \mu m_p}{R}$. \\

\begin{align}
L_{fusion} \pt \frac{M^{6/5}}{R^{7.5}}
\end{align}

If you change the $L$ of the star, $T$ and $\rho$ must change to accommodate. In steady state:

\begin{align}
L_{fusion} \pt \frac{M^{6/5}}{R^{7.5}} & \pt \frac{ M^{5.5} }{\sqrt{R}}\\
R & \pt M^{1/7}\\
T_c &\pt M^{6/7}\\
L & \pt M^{5.4}\\
L & \pt T_{eff}^4~ \pt 4\pi R^2\sigma T^4~, \text{but R dep is so weak it's estimated to be constant}
\end{align}

\begin{align}
M \uparrow~,T_c \uparrow \sim M^{6/7}\\
\epsilon_{pp} \pt T^{4.5}~, \epsilon_{CNO} \pt T^{20}
\end{align}

Stars that are more massive than the sun are very $T$ dependent. \\

For $M \geq M_\odot$: $\kappa \sim \kappa_T \sim$ constant. $\gamma$ still dominate $E$ transport. CNO cycle is dominant mechanism and $P_{gas}$ dominates. $L_{rad} \pt M^3$.

\begin{align}
L_{fusion}&  = \int \epsilon_{CNO} dM_r\\
 & \sim \epsilon_{CNO}M~, \epsilon_{CNO} \pt \rho T_c^{20}\\
 \epsilon_{CNO} \pt \frac{M^{21}}{R^{23}}\\
 L_{fusion} \sim L_{rad}\\
  \frac{M^{22}}{R^{23}} \pt M^3\\
  R \pt M^{19/23} \pt M^{.8}\\
  T_c \pt M^{.2}
\end{align}

\begin{align}
L & = 4 \pi R^2 \sigma T_{eff}^4\\
L & \pt R^2 T_{eff}^4\\
R^2  & \pt M^{1.6} \pt L^{1/2}\\
L & \pt M^3\\
   & \pt L^{1/2}T_{eff}^4\\
L^{1/2} & \pt T_{eff}^4\\
L & \pt T_{eff}^8\\
T_{eff} & \pt M^{3/8}
\end{align}

A huge change in L corresponds to a small change in $T$. Only for stars with masses slightly more than the sun. \\

For $M \geq 1.2 M_\odot$, they have convective cores and $\gamma$ transport energy on outer part of star. Reverse of our sun. Convection sets in if $\frac{ds}{dr} < 0$. $\frac{ds}{dr}$ is implied by $\gamma$ transport of energy. You can then use Radiative diffusion equation to see if $\frac{ds}{dr}<0$. i.e. Convection sets in if $\frac{d \ln T}{d \ln P} > \frac{\gamma -1}{\gamma}$. $\gamma$ is the one for photons, not of the particles convecting. 

\begin{align}
\frac{d \ln T}{d \ln P}  = \frac{1}{4} \frac{P}{P_{rad}}\frac{L}{L_{edd}}\frac{L_r/L}{M_r/M}~,L_{edd} = \frac{4 \pi G M_c}{\kappa}
\end{align}

$P$ is the total pressure. This tells us convection sets in if $\frac{1}{4} \frac{P}{P_{rad}}\frac{L}{L_{edd}}\frac{L_r/L}{M_r/M} > \frac{2}{5}$. Recall for CNO, $\epsilon \pt \rho T^\beta$. At almost all $r$, $L_r \approx L$. For CNO-dominated stars, only 1\% of star's mass fuses. TINY! It's this enormous flux that originates so close so the core that it drives convection. $\frac{M_r}{M} < \frac{5}{8} \frac{P}{P_{rad}}\frac{L}{L_{edd}}$, then convection sets in. We're interested where $P \approx P_{rad}$. We want to know $\frac{P}{P_{rad}}$ and $\frac{L}{L_{edd}}$. $L \pt M^3$ and $L_{EDD} \pt M$ so $\frac{L}{L_{edd}} = 4.5 \times 10^{-5} \left( \frac{M}{M_\odot} \right)^2$. \\

In the sun, $r \sim 0 \rightarrow .5 R_\odot$, $\frac{P_{gas}}{P_{rad}} \sim 3000$.

\begin{align}
P_{gas} &\pt \rho T \pt \frac{M^2}{R^4}\\
P_{rad}& = \frac{1}{3}a T^4 \pt \frac{M^4}{R^4}\\
\frac{P_{gas}}{P_{rad}} &\pt M^{-2}\\
\frac{P_{gas}}{P_{rad}} &\approx 3000 \left( \frac{M}{M_\odot} \right)^{-2}
\end{align}

Convection sets in if $\frac{M_r}{M} \leq 0.1$. \\

Lifetime of MS star $\approx \frac{E_{nuc}}{L}$. 

\begin{align}
L = L_\odot \left( \frac{M}{M_\odot} \right)^{3.5}\\
E_{nuc} &= N_p E\\
&  \approx  .1 \frac{M}{m_p}~\cdot 7 ~\text{MeV}\\
\frac{E_{nuc}}{L} \approx 10^{10} \left( \frac{M}{M_\odot} \right)^{-2.5}
\end{align}

ANY star with a mass less than .85$M_\odot$ is still fusing after 13.7 billion years. For $M \sim 30 M_\odot$, $t_{MS}$ is about $10^6$ years. Massive stars live and die where they are born. 

%%%%%%%%%%%%%%%%%%%%%%%%%%

\chapter{QM Effects in Stars}

\begin{center}
\textbf{\begin{huge} October 18, 2011\end{huge}}
\end{center}

\section{Min and Max Masses of Stars}

A star is an object held together by it's own gravity; undergoes H fusion into He. Doesn't matter if it's fused in the past, it's still a star, just at a dif phase in it's life. In the present day universe, stars have $M$ between $.08 M_\odot  \leq M \leq 100-200 M_\odot$. The fact that stars under $.08M_\odot$ don't undergo fusion is well understood, due to the QM nature of stars. The fact that stars above $100-200 M_\odot$ don't fuse, however, isn't well understood. Best guess it has something to with $P$ dominated by $P_{rad}$.

\subsection{Lower Limit}

Due to the QM properties of the gas in stars. We've treated the gas in stars as an ideal, classical gas. $P  = nkT + \frac{1}{3}aT^4$. What is required for the aforementioned equation to be valid? 1) QM degeneracy $P$ is small, 2) at typical distance, not interacting with itself. QM nature is important when:

\begin{align}
\lambda \gtrsim \text{ distance between particles} \sim n^{-3}\\
p_{th} = mv_{th} \approx \sqrt{mkT}\\
\lambda = \frac{h}{p} = \frac{h}{\sqrt{mkT}}\\
\end{align}

The QM nature is important if $\lambda \geq n^{-3} $. $n \geq \lp \frac{mkT}{h^2} \rp^{3/2}$. 

\begin{align}
n_Q &\equiv \text{ quantum density}\\
&= \lp \frac{2 \pi m k T}{h^2} \rp ^{3/2} = n_Q(T)\\
n &\geq n_Q~,\text{ then QM is important}
\end{align}

If $n_q \pt \m^{3/2}$, then the first particles you have to worry about are the electrons, not the protons. Electrons have a lower mass, but what about photons? \\

At the center of the sun, density of electrons is actually close to the quantum density. So what we've been doing so far has been wrong? If we think about stars of different masses, we need to look at how the $T$ and $\rho$ change with $M$. 

\begin{align}
R \pt M^{3/4}\\
T_C \pt \frac{M}{R} \pt M^{1/4} \downarrow \text{ as } M \downarrow~, n_Q \downarrow \text{ too}\\
\rho \pt \frac{M}{R^3} \pt \frac{M}{M^{9/4}} \pt M^{-5/4}~, n \uparrow \text{ as } M \downarrow\\
\end{align}

QM becomes increasingly important for low mass stars. It's the electrons we worry about, specifically. Distribution function of particles isn't really MB, but something more general.\\

\begin{align}
n(p)= \frac{2/h^3}{e^{(E_p - \mu)/kT} \pm 1}~, E_p = p^2 c^2 + m^2 c^4,\mu = \text{ chemical potential} = \frac{\delta E}{ \delta N} \Bigl\lvert_{S,V}
\end{align}

$\pm$ relates to: "$+$" obeys Fermi-Dirac Statistics and "$-$" obeys Bose-Einstein Statistics. We want Fermi statistics because electrons lie there. $n(p)$ is the number density defined to be: 

\begin{align}
n= \int d^3 p \cdot n(p)~,
\end{align}

where $n(p)$ is the number of particles per unit volume in $d^3x \pt d^3p$. Real space and in momentum space. Now we're going to focus on Fermions...

\subsection{Fermions}

\begin{align}
n(p)= \frac{2/h^3}{e^{(E_p - \mu)/kT} + 1}
\end{align}

Let's consider a fully degenerate gas. i.e. QM nature is very very important. This is at the limit $n \gg n_Q$ where $T \rightarrow 0$. What is the limit of the DF as $T \rightarrow 0$? 

\begin{align}
e^{\pm \text{ big \#}}
\end{align}

$n(p) \simeq 0 $ if $E_p > \mu$ and $n(p) \simeq \frac{2}{h^3} $ if $E_p < \mu$. This is only in the QM limit though. But! Stars never reach $T = 0$, so what do we really mean? When we say the T is 0, we mean the thermal energy is small compared to something. More precisely, $\mu \gg kT$. \\

We can just calculate the number density $n$ of fermions, $n = \int n(p) \cdot d^3p$. This is either an integral of 0 or a constant. Define: $p_F$ as the Fermi Momemutm

\begin{align}
p_F = p \text{ such that }E_p = \mu
\end{align}

Then, 

\begin{align}
n &= \int_{0}^{p_F} \frac{2}{h^3}d^3p\\
&= \frac{8 \pi}{h^3} \int_0^{p_F} p^2 dp\\
&= \frac{8 \pi}{3h^3} p_F^3\\
\Aboxed{p_F &= \lp \frac{3h^3}{8 \pi} \rp^{1/3} n^{1/3}}~,\text{ this is true in both NR and R limits}
\end{align}

Let's assume NR fermions:

\begin{align}
E_p &= \frac{1}{2}mv^2 = \frac{p^2}{2m}\\
E_F &= \frac{p_F^2}{2m}\\
\Aboxed{&= \lp \frac{3h^3}{8 \pi} \rp ^{2/3} \frac{n^{2/3}}{2m} }
\end{align}

This can also be described as $\mu$. $\boxed{\mu = E_F}$. No matter how close to $T=0$ you get, particles will still move!

\begin{list}{}{}
\item gas density $n$
\item typical dist between particles $\sim n^{-1/3}$
\end{list}

If $\lambda \geq n^{-1/3}$, particles seem to blur together. But the P.E.P says that particles needs to be in separate discrete states. This blurring isn't possible for fermions that obey the P.E.P. If you obey the P.E.P, you have to have a $\lambda \leq n^{-1/3}$, even if $T \rightarrow 0$. As you get closer to $T \rightarrow 0$, then $\lambda \geq n^{-1/3}$, but at some point, P.E.P. forces $\lambda \leq n^{-1/3}$. 

\begin{align}
\frac{h}{p}\sim n^{-1/3} \rightarrow p \sim hn^{-1/3} \sim p_F
\end{align}

The $p_F$ is largely determined by $\lambda$. Whether particles are R or NR is irrelevant, they can still be QM in both situations. If we assume things are NR, $E_F \sim \frac{p_F^2}{2m		} \sim \frac{h^2}{2m} n^{2/3} \equiv \mu$. The flux you radiate doesn't care about the Fermi nature of the electrons. When we say $F = aT^4$, we mean $T$ to be $T$, not kinetic energy, not thermal energy, just $T$. With QM, we're saying that even with a low $T$, we can still have a high $p$ and $E$. \\

What we really want to understand is the structure of low mass stars. What is the pressure produced by these particles? This is what's going to compete with gas or radiation pressure. Remember

\begin{align}
P &= \frac{2}{3} \epsilon\text{ for NR} \\
P &= \frac{1}{3} \epsilon \text{ for R}
\end{align}

Our guess would be for the NR case:

\begin{align}
P  &\sim n E_F~,\text{ number of particles per unit volume time energy per particle}\\
P &\sim \frac{h^2}{2m}n^{5/3}
\end{align}

\begin{align}
\epsilon &= \int \frac{p^2}{2m}n(p) d^3p\\
&= \int_0^{p_F} \frac{p^2}{mh^3}4 \pi p^2 dp \\
&= \frac{4 \pi}{mh^3} \int_0^{p_F} p^4 dp\\
\Aboxed{\epsilon &= \frac{4 \pi}{5} \lp \frac{3}{\8 \pi} \rp^{5/3} \frac{h^2 n^{5/3}}{m}}
\end{align}

\begin{align}
\Aboxed{P &= \frac{h^2}{5m} \lp \frac{3}{8 \pi} \rp^{2/3} n^{5/3}}~\text{ NR QM degeneracy pressure of fermions}\\
&\pt \frac{n^{5/3}}{m}~,\text{ lowest mass particles dominates pressure (in our case, $e^-$)}\\
&\pt \frac{n^{5/3}}{m}~,\text{ $n=3/2$ polytrope!}
\end{align}

Since $P \pt n^{5/3}$ it doesn't mean the star is convective although the opposite is true. It's totally unrelated. Elliot thinks. We assumed a NR gas of fermions, now how about R? $p_F$ is so large that the velocity approaches the speed of light. Can't use $P &= \frac{2}{3} \epsilon$, must use $P &= \frac{1}{3} \epsilon$.

\begin{align}
\epsilon \sim n \cdot p_F c~,p_fc = E_F\\
p_F \sim hn^{1/3}\\
P _{\text{R Degnererate Gas}}&\sim hn^{4/3}c\\
&\sim \frac{hc}{4} \lp \frac{3}{8 \pi} \rp^{1/3} n^{4/3}~,\text{ only depends on $p$, not $p$ and $m$.}
\end{align}

This is an example of a $n=3$ polytrope. 



\end{document}