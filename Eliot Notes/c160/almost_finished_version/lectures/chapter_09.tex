
\chapter{Star Formation}

\begin{center}
\textbf{\begin{huge} September 22, 2011\end{huge}}
\end{center}

\section{Star Formation}

Gas in galaxies comes in multiple "phases". It's still a gas, just a broad range with particular characteristics. They have different $\rho$ \& $T$ with comparable $P$. Hot, low $\rho$ gas is mostly in the form of hot ISM. Stars form from \textbf{cold molecular clouds}. What are the conditions for a cold molecular gas cloud to collapse? 

\begin{align}
|U| &\geq |K|\\
\frac{GM}{R^2} &\gtrsim \Bigl\lvert \frac{dP}{dr} \Bigl\lvert\\
\text{self gravity of cloud} &\gtrsim \frac{3}{2}NkT\\
&\approx \frac{M}{m_p}kT
\end{align}

If $M \gtrsim \frac{RkT}{Gm_p}$, then it will collapse. 

\begin{align}
\rho &\approx \frac{M}{R^3}\\
R &\sim \lp \frac{M}{\rho} \rp^{1/3}\\
M^{2/3} &\gtrsim \frac{kT}{Gm_p \rho^{1/3}}\\
\Aboxed{M_J &\geq \lp \frac{k}{Gm_p} \rp^{3/2} \frac{T^{3/2}}{\sqrt{\rho}}}~,M_J = {\text{ Jeans Mass}}\\
\frac{GM^2}{R} &\geq \underbrace{\frac{MkT}{m_p}}_{c_s^2} \\
\text{if } \frac{1}{\sqrt{G\rho}} &\leq \frac{R}{c_s}~,\text{ then if $t_{FF} < t_{sound}$, and it will collapse}
\end{align}

Stars are more prone to collapse if they have lower $T$ and higher $\rho$. Stars form from cold molecular clouds because they are the most unstable.

\begin{align}
M_J &\approx 50 M_\odot \frac{\lp \frac{T}{10K}\rp^{3/2}}{\lp \frac{\text{n}}{100} \text{ cm}^{-3} \rp^{1/2}}\\
R_J &= \lp \frac{M_J}{\rho} \rp^{1/3}\\
&\approx 3 \text{ pc} \frac{(\frac{T}{10K})^{1/2}}{(\frac{\text{n}}{100}\text{ cm}^{-3})^{1/2}}
\end{align}

If a star has $M>M_J$ and $R<R_J$, then it will collapse. The collapse time is $\sim \frac{1}{\sqrt{G\rho}} \sim 10\text{ Myr} \lp \frac{n}{100}\text{ cm}^{-3} \rp^{-1/2}$. Why don't we have tons of $50M_\odot$ stars? In reality, most of them are roughly $0.3M_\odot$. 

\begin{align}
\rho a &= -\frac{dP}{dr}- \rho \frac{GM}{R^2}\\
&\sim \frac{P}{R} - \frac{GM^2}{R^5}\\
&\pt \frac{nT}{R}\\
&\pt \frac{MT}{R^4}
\end{align}

\section{The Collapse Process}

Initially, the gas cools rapidly and since photons easily escape cloud, $T \sim$ roughly constant, isothermal at around $10K$. $P \pt \frac{M}{R^4}$, \& gravity $\pt \frac{M^2}{R^5}$. As radius decreases, gravity dominates. There is clearly no halt to the collapse while $M=$ constant. What's happening to $M_J \pt \frac{T^{3/2}}{\sqrt{\rho}}$? So during this process, $M_J \downarrow, \rho \uparrow$. Little regions within the cloud collapse on themselves, so what determines the mass of the small cloudlets? \\

\noindent Now the cloudlets are sufficiently dense so that energy can't escape. The cloud becomes adiabatic, $\kappa \uparrow, l \downarrow, T \uparrow$. The random walk time of the photons is now longer than $t_{ff}$; for adiabatic:

\begin{align}
P &\pt \rho^\gamma\\
\rho &\pt \rho^\gamma\\
T &\pt \rho^{\gamma-1} = \rho^{2/3}\\
T &\pt \frac{M^{2/3}}{R^2}
\end{align}

Now, plug in the above expression for $T$ into $P \pt \frac{MT}{R^4}$ to get $P \pt \frac{M^{5/3}}{R^6}$. Yay! No more runaway collapse. Now, $\frac{dP}{dr} \sim -g\rho$ and we're back in HE. $M_J \uparrow, T\pt \rho^{2/3}, M_J \pt \rho^{1/2}$, which increases as the cloud collapses.\\

The cloud is now in HE, but $R \gg R_\odot$. There exists a luminosity of the cloud from simply losing $E$. This contraction will (eventually) lead to fusion.

\begin{align}
E_{TOT} = \frac{U}{2}\\
L = -\frac{dE}{dt} = -\frac{1}{2} \frac{dU}{dt}\\
U = -\frac{GM^2}{R}\\
L = \frac{GM^2}{2R^2} \Bigl\lvert \frac{dR}{dt} \Bigl\lvert
\end{align}

If the cloud is radiative, $L \pt M^3$. If the cloud is convective, $L \simeq 0.2 L_\odot \lp \frac{M}{M_\odot}\rp^{4/7} \lp \frac{R}{R_\odot}\rp^2$. 

\begin{align}
L_{conv} = \frac{1}{2}\frac{GM^2}{R^2} \Bigl\lvert \frac{dR}{dt} \Bigl\lvert\\
\Aboxed{\frac{R}{R_\odot} &\approx \lp \frac{M}{M_\odot}\rp^{1/2} \lp \frac{t}{2 \times 10^7 \text{ years}} \rp^{-1/3}}\\
R &\pt t^{-1/3}
\end{align}

Plug in the above expression for $\frac{R}{R_\odot}$ into $L \simeq 0.2 L_\odot \lp \frac{M}{M_\odot}\rp^{4/7} \lp \frac{R}{R_\odot}\rp^2$ to get:

\begin{align}
L&\approx 0.2 L_\odot \lp \frac{M}{M_\odot}\rp^{11/7} \lp \frac{t}{2 \times 10^7 \text{ years}}\rp^{-2/3}\\
L &\pt t^{-2/3}
\end{align}

As the cloud contracts, $L \downarrow, R \downarrow$. For a fully convective cloud, $L \uparrow$ if $R \uparrow$. They all have similar $T_{eff}$. $L \sim 4 \pi R^2 T_{eff}^4$, $L \pt R^2$, which is what we have.
