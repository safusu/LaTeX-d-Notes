
\chapter{Thermonuclear Fusion}

\begin{center}
\textbf{\begin{huge} September 27, 2011\end{huge}}
\end{center}


\section{Thermonuclear Fusion}

Nuclei-\\
\begin{list}{$\circ$}{}
\item Z - proton \# 
\item A - mass \#
\item N - \# of neutrons (N= A-Z)
\item Proton mass = $m_p c^2 = 938.259$ MeV
\item Neutron mass = $m_n c^2 = 939.553$ MeV
\item 1 MeV 1.6 $\times 10^{-6}$ ergs
\item Isotope - same Z (Carbon 12, 14 are isotopes)
\item Isobar - same atomic \# (Carbon 14, Nitrogen 14)
\item size of nucleus $\sim$ 1.3 $A^{1/3}$ fm  ($10^{-15} m = 10^{-13} cm$)
\end{list}

Free n can $\beta$ decay
\begin{equation}
n \rightarrow p + e^- + \overline{\nu_e}~, \text{will decay in something like 900 s}
\end{equation}

Tells us nuclei have constant density... so we take the...

\begin{align}
\rho & = \frac{A m_p}{\frac{4 \pi}{3}r_n^3} \\
& = \frac{m_p}{\frac{4 \pi }{3}1.3 ~\text{fm}}\\
& = 10^{14} \text{g cm}^{-3}
\end{align}

Size set by strong force. Falls off quickly for $r > r_n$. \\

Some of the forces we'll take about are the long-range forces. (Gravity, E\&M...) The particle transmitting the force has to be massless. IN E\&M, it's the photon. In gravity, it's the graviton. Its the fact that these particles are massless lets us have this long range force. For TN, we have the strong force... but the particle which mediates the force has a mass. \\

Finite rest mass $\rightarrow$ short range force\\ 

\begin{eqnarray}
\Delta E \Delta t \sim \hbar \rightarrow \Delta t \sim \frac{\hbar}{E}\\
d \sim c \Delta t\\
E \sim \frac{\hbar c}{d} \sim \frac{ 197 ~\text{MeV fm}}{1 ~\text{fm}} \sim 200~ \text{MeV}
\end{eqnarray}

This particle turns out to be the pion. $m_\pi c^2 \sim 140$ MeV. \textbf{When are the nuclei stable? }

\subsection{Nuclear Stability}
Not every Z \& A are stable. There's a region of stability for low Z, elements that have Z = N are stable and at higher Z, N $\geq$ Z are stable. But why? Let's use shell nucleus (remember n \& p, like $e^-$) \textit{Pauli Exclusion Principle!} Use the shell mode to build out. \\

The main difference between the shell model for electron and the shell model for the nucleus is that... it's different. There are two energy levels, one for each proton and neutron, for example. n can $\beta$ decay, $n \rightarrow p + e^- + \overline{\nu_e}$. The neutrons move to lower energy states and into protons. Sometimes you can also reach total lower energy if you turn a spare proton in its own energy level into a neutron to pair with one that's alone. $ p \rightarrow n + e^+ + \nu_e$. Basically, you want to put things in the lowest energy level. This process favors Z $\sim$ N. Now this breaks down... but why? \\
Once you get to massive nuclei, the Coulomb repulsion starts kicking in. The EM repulsion wants to fight back. More neutrons means more strong force, which lets you hold together the protons which are repelling. If you want to build stable nuclei, you're going to need to add more neutrons than protons. How much energy is holding nuclei together? (Binding energy) Held together by strong force. 

\begin{align}
E_\text{nuc} & = Zm_pc^2 + N m_n c^2 -E_b = m_\text{nucleus}c^2~,\\
\frac{E_b}{A} & = \text{binding energy per nucleon}\\
& \approx 8~\text{MeV (peaks at Fe 56 and secondly at Ni 62)}
\end{align}

For $^4$He, $\frac{E_b}{A} \approx 7$ MeV. 

\subsection{Fusion}
light + light $\rightarrow$ heavier + energy\\
Happens for $A \leq 60$. If you try to do lead + lead, don't get anything out. Less bound + less bound $\ra$ more bound + energy, to put it better. What sets the nuclear energy scales? \\

Nuclear energy scale:\\
Coulomb repulsion of proton:\\
\begin{align}
E & \sim \frac{(Ze)^2}{r}~,\\
& \sim \frac{(Ze)^2}{A^{1/3} ~ \text{fm}} \sim 1 ~ \text{MeV}
\end{align}

What's the Fermi energy of the nucleon?

\begin{align}
E_F & \sim \frac{3}{5} \frac{p_F^2}{2 m_p}~, p_F \sim \left( \frac{3 n^{1/3}}{8 \pi} \right) h\\
& = 25 ~\text{MeV}
\end{align}

Now that we've done nuclear physics, doing order of magnitude for fusion.

\subsection{Order of Magnitude Estimates for Fusion}

What's the dominant reaction (in the sun)? More or less, it's $4 \text{p} \ra ^4$He. The BB gives lots of H. The binding energy of Helium is about 28.3 MeV. How much energy can we get out of the sun if all of its H turns into He? 

\begin{align}
E_\odot = 28~\text{MeV} \left( \frac{M_\odot}{4m_p} \right) = 10^{52} ~\text{ergs}\\
t_{nuc} \sim \frac{E_\odot}{L_\odot} \sim \frac{10^{52}~\text{ergs}}{4 \times 10^{33} ~\text{ergs}} \sim 3 \times 10^{18} ~ \text{s} \sim 10^{10} ~ \text{yr}
\end{align}
However,  only fuses H $\ra$ He in the central 10\% of the sun (by mass). \textbf{Why is fusion hard?}

\subsection{Why is Fusion Hard?}
The impediment is Coulomb repulsion. We have to exert some energy to push them together then let strong force take over and fuse them together. \\

Let's say two protons are separated by a distance r.

\begin{align}
E = \frac{1}{2} \mu v^2 + \frac{e^2}{r}~, E \approx kT
\end{align}

When the proton are closest together, their $v=0$. How much energy do we need to get the $p$ nuclearly close together? 

\begin{align}
E \approx \frac{v^2}{r_c} \approx kT~, r_c \sim \frac{e^2}{kT}
\end{align} 

We want $r_c < r_n$. $\ra$ $kT \geq \frac{e^2}{r_n} \ra T \geq 10^{10}$K. But the central $T_\odot$ is only $10^7$ K! Did we do something wrong? Yeah! Why? We didn't use QM.\\
How do we estimate if QM is important? \\

deBr\"oglie Wavelength:
\begin{align}
\lambda \sim \frac{h}{p} \sim 10^{-10} \left( \frac{T}{10^7 ~\text{K}} \right)^{-1/2}~ \text{cm}
\end{align}

If $\lambda \gg r_n$, then we cannot ignore QM. How high does our energy have to be to get over the barrier? What barrier? Oh, you don't have a graph.\\

There's some finite probability that we can get through the potential hump where $E \ll \frac{e^2}{r_n}$ and reach $r_n$ \& feel strong force. Tunneling is important when:

\begin{align}
r_c \sim \lambda ~.
\end{align}

What is $r_c$? It's defined as $\frac{Z_1 Z_2 e^2}{r_c} \sim \text{kT} \sim \frac{p^2}{2m}$. 

\begin{align}
r_c & = \frac{Z_1 Z_2 e^2 (2m)}{p^2} \sim \lambda \sim \frac{h}{p}~, p = \sqrt{\text{kT} m_p}\\
\text{kT} & \sim \frac{4 Z_1^2 Z_2^2 e^4 m_p A}{h^2}\\
\text{T} & \sim 3 \times 10^7 Z_1^2 Z_2^2 ~\text{K}~, \text{which is more on order of magnitude of central T}
\end{align}

\section{Big Picture}
Took HE and found kT $=\frac{GM \mu m_p}{3R}$. Took radiative diffusion $L \pt M^3$ and fusion: T $\sim~10^7$ K. KH contraction of $M_\odot$ of gas: $R \gg R_\odot$, $T \ll T_\odot,R \downarrow,T \uparrow, L_{\text{fusion}} = L_{\text{radiative diffusion}}$. 

