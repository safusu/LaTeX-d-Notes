\chapter{White Dwarfs}

\begin{center}
\textbf{\begin{huge} November 10, 2011\end{huge}}
\end{center}

\section{Non-Relativistic White Dwarfs}

Supported by NR electron degeneracy pressure.

\begin{align}
P &=  \frac{h^2}{5m_e}\lp \frac{3}{8 \pi} \rp^{2/3} n_e^{5/3}\\
P &\sim E_F n~,E_F = \frac{p_F^2}{2m}~,p_F \sim hn^{1/3}
\end{align}

\begin{align}
n_e &= \frac{\rho}{\mu_e m_p}\\
P &= K\rho^{5/3}~,K = \frac{h^2}{5m_e} \lp \frac{3}{8\pi} \rp^{2/3} \lp \frac{1}{\mu_e m_p}\rp^{5/3}
\end{align}

For a $n=\sfrac{3}{2}\textrm{ polytrope}$,

\begin{align}
P_c &= 0.77 \frac{GM^2}{R^4}\\
\rho_c &= 6\langle \rho\rangle = \frac{GM}{\frac{4}{3} \pi R^3} = \frac{9}{2\pi} \frac{M}{R^3}\\
P_c &= K\rho_c^{5/3}\\
0.77 \frac{GM^2}{R^4} &= K \lp \frac{9}{2\pi}\rp^{5/3} \frac{M^{5/3}}{R^5}\\
\Aboxed{R &\pt M^{-1/3}}
\end{align}

For degenerate objects, more massive objects are physically smaller.

\begin{align}
R &= 2.34 \frac{K}{G} M^{-1/3}\\
\Aboxed{&= 0.04R_\odot \mfrac^{-1/3}\mu_e^{-5/3}}
\end{align}

So for C/O, 

\begin{align}
\mu_e &= 2\\
\Aboxed{R &\simeq 0.013 R_\odot \mfrac^{-1/3} \lp \frac{\mu_e}{2} \rp^{-5/3}}
\end{align}
For $M\sim M_\odot$, $R \sim 10^9$ cm $\sim R_\oplus$.

We assume the electrons were NR in the above equations. We know that the $p_F\sim n^{1/3}h$.
\begin{align}
\langle \rho \rangle = \frac{M}{\frac{4}{3} \pi R^3}\approx 6 \times 10^5 \mfrac^2\textrm{ gm cm}^{-3}\\
\rho_c = 6\langle \rho \rangle \approx 4 \times 10^6 \mfrac^2 \textrm{ gm cm}^{-3}
\end{align}
\begin{align}
p_F \sim 3 \times 10^{-17} \mfrac^{2/3} \textrm{ gm cm s}^{-1}
\end{align}
We can use the simple relation $p_F = v_em_e$ to get $v_e \approx 4 \times 10^{10} \mfrac^{2/3}$ cm/s. WHAAAT. As $M$ increases, $v_e$ increases and $e^-$ become increasingly relativistic. 

\section{Relativistic White Dwarf Stars}

\begin{align}
P &\sim E_F n\\
E_F &= \underbrace{\frac{p_F^2}{2m}}_{\textrm{non-relativistic}} \textrm{ or } \underbrace{p_Fc}_{\textrm{relativistic}}
\end{align}

If $e^-$ are completely relativistic, $E_F \gg m_ec^2$ and so we can ignore rest mass energy. 

\begin{align}
E_F &= p_Fc\\
p &\sim hcn_e^{4/3}\\
\Aboxed{P &= \frac{hc}{4} \lp \frac{3}{8\pi}\rp^{1/3}n_e^{4/3}}
\end{align}

Now let's do it again!

\begin{align}
n_e& = \frac{\rho}{\mu_e m_p}\\
P &= K\rho^{4/3}~,K = \frac{hc}{4} \lp \frac{3}{8\pi} \rp^{1/3} \lp \frac{1}{\mu_e m_p}\rp^{4/3}
\end{align}
This is a $n=3$ polytrope.
\begin{align}
P_c &= 11 \frac{GM^2}{R^4}\\
\rho_c &= 54.2 \langle \rho \rangle \\
&= 12.9 \frac{M}{R^3}\\
P_c &= K\rho_c^{4/3}\\
11 \frac{GM^2}{R^4} &= K30\frac{M^{4/3}}{R^4}\\
\Aboxed{M &\approx 4.6 \lp \frac{K}{G} \rp^{3/2}}\\
M &\approx 0.2 \frac{(hc)^{3/2}}{(\mu_e m_p)^2G^{3/2}}\\
\Aboxed{M &\approx 1.45 \lp \frac{\mu_e}{2} \rp^{-2} \ms = M_{ch}}
\end{align}

But what does this mean? Let's look at Hydrostatic Equilibrium:

\begin{align}
\frac{dP}{dr} &= -\rho \frac{GM_r}{r^2}~,\rho \sim \frac{M}{R^3}\\
P &\sim \frac{GM^2}{R^4}\\
P &\pt \rho^{4/3} \pt \frac{M^{4/3}}{R^4}
\end{align}

For a NR electron gas:
\begin{align}
P &\pt \rho^{5/3} \pt \frac{M^{5/3}}{R^5}\\
\frac{M^{5/3}}{R^5}&\pt \frac{M^2}{R^4}\\
R &\pt M^{-1/3}
\end{align}
\begin{align}
\frac{M^{4/3}}{R^4} &\pt \frac{M^2}{R^4}~M\uparrow\textrm{ gravity $>$ pressure}
\end{align}

So as $M \uparrow$, $R\downarrow$ in Hydrostatic Equilibrium. As M gets larger and larger, there will be some maximum mass above which we can't have pressure balance gravity. Hydrostatic Equilibrium will fail. $M_{ch}$ is the maximum mass for objects supported by electron degeneracy pressure. 

\section{The Life of a White Dwarf}

White Dwarf stars are born hot and luminous but fade away in time. There's a reservoir of thermal energy that it's born with that it radiates slowly over time. Energy transport  in a white dwarf is dominated by thermal conduction (also true in neutron stars). This is different from stars like the sun where energy transport by photons is much much more effective than conduction.

\begin{align}
F &= -\kappa \nabla T~,\kappa = \frac{1}{3}vl\frac{dU}{dT}~, \frac{dU}{dT}= \textrm{specific heat of particles}
\end{align}

For a normal star, $v = v_{th} = \sqrt{\frac{kT}{m_e}}$
\begin{align}
\frac{dU}{dT} \sim nk~,U = \frac{3}{2}nkT~,l=l_c\textrm{ coulomb scattering}
\end{align}
but in a WD,
\begin{align}
v &\sim \frac{p_F}{m_e} \sim c\\
U &= nE_f~,\textrm{ but if it was the case, then } \frac{dU}{dT} = 0
\end{align}
Instead, we have to add a term to $nE_F$ to correct:
\begin{align}
U &\sim nE_F + \theta (kT)^2\\
\frac{dU}{dT} &\approx nk \underbrace{\frac{kT}{E_F}}_{\ll1}
\end{align}

Scattering in a white dwarf is still dominated by $l_c$, but the energy of an electron is no longer dependent on temperature. We used to have $\frac{e^2}{b} \sim E_i$, where in the classical case, $E_i = kT$ and in the white dwarf case, $E_i = E_F$. 

\begin{align}
\Aboxed{\sigma_c &\sim \pi b^2 \sim \frac{ \pi e^4}{E_F^2}}
\end{align}

In a white dwarf where the $E_F \gg kT$, $\sigma_c \ll $ normal star and therefore $l_c = \frac{1}{n\sigma_c} \gg $ normal star. Putting this all together,

\begin{align}
\kappa_{\textrm{degen}} = 10^3 \cdot T_8^{-3/2}\kappa_{\textrm{classical}}(T)
\end{align}

This is why electron conduction is much more important than photon energy transport. 

\begin{align}
\kappa_{\textrm{degen}} \approx \frac{k^2h^3Tn_i}{Z_i^232e^4m_e^2}~,n_i = \textrm{ ion-$e^-$ scattering}
\end{align}

Let's imagine a $T$ profile just after fusion: 

%Insert plot here 

It'll try to make temperature uniform and after a little while, the $T_c$ is pretty much constant. The time to become isothermal is roughly $10^6 \sim 10^7$ years. Over time, the white dwarf cools by radiating initial thermal energy away. We just said that the energy of an electron is proportional to $E_F\lp 1 + \lp \frac{kT}{E_F}\rp^2 \rp$. C/O isn't degenerate, so there, $E = kT$. The important note is that it's mostly the thermal energy of the ions that's being radiated. 

\begin{align}
L_{\textrm{WD}} \approx 5L_\odot \mfrac \lp \frac{T_c}{10^8 \textrm{ K}} \rp^{7/2}~
\end{align}

Remember we use $T_8$ because it's about what the C/O core is born at.
\begin{align}
\frac{T_c}{10^8\textrm{ K}} &\approx \lp \frac{t}{10^6\textrm{ yr}} \rp\\
L &\approx 5 L_\odot \mfrac \lp \frac{t}{10^6 \textrm{ yr}} \rp^{-7/5}
\end{align}
\begin{align}
L &= 4\pi R^2 \sigma T_{eff}^4\\
T_{eff} &\approx 58,000 \lp \frac{L}{L_\odot} \rp^{1/4} \textrm{ K}\\
\Aboxed{T_{eff} &\pt t^{-0.35}}
\end{align}

A 1 Gyr white dwarf would have:
\begin{align}
L &\sim 3 \times 10^{-4} L_\odot\\
T_{eff} &= 9000\textrm{ K}
\end{align}

A 10 Gyr white dwarf would have:
\begin{align}
L &\sim 10^{-5} L_\odot\\
T_{eff} &= 3000\textrm{ K}
\end{align}

