\documentclass[10pt,a4paper,final]{article}
\usepackage[latin1]{inputenc}
\usepackage{amsmath}
\usepackage{amsfonts}
\usepackage{amssymb}
\author{Jeren}
\title{HW \#5}
\begin{document}
Problem 1a:

\begin{align}
L & \approx L_\odot \left( \frac{M}{M_\odot} \right)^{11/2} \left( \frac{R}{R_\odot} \right) ^{-1/2}\\
L & \approx \frac{GM^2}{2Rt}\\
M^{11/2} R^{-1/2} & \propto \frac{M^2}{Rt}\\
R^{1/2} & \propto \frac{M^2}{M^{11/2}t}\\
R & \propto \left( \frac{M^{2-{11/2}}}{t} \right)^2\\
R & \propto \left( \frac{M^{-7/2}}{t} \right)^2\\
R & \propto M^{-7}t^{-2}
\end{align}

\begin{align}
L & \propto M^{11/2}R^{-1/2}\\
L & \propto M^{11/2}(M^{-7}t^{-2})^{-1/2}\\
L & \propto M^{11/2}M^{7/2}t\\
L & \propto M^9 t
\end{align}

\begin{align}
L & \propto T^4R^2\\
T^4R^2 & \propto M^9 t\\
T^4 & \propto M^9 R^{-2} t\\
T^4 & \propto M^9 (M^{-7}t^{-2})^{-2} t\\
T^4 & \propto M^9 (M^{14}t^4)t\\
T & \propto (M^{23}t^5)^{1/4}\\
T & \propto M^{23/4} t^{5/4}
\end{align}

Problem 1b:

\begin{align}
E_\gamma = E_{conv}\\
L_{conv} = 0.2 L_\odot \left( \frac{M}{M_\odot} \right)^{4/7} \left( \frac{R}{R_\odot} \right)^2\\
L_{rad} \approx L_\odot \left( \frac{M}{M_\odot} \right)^{11/2} \left( \frac{R}{R_\odot} \right)^{-1/2}\\
\end{align}

set them equal to each other

\begin{align}
0.2 L_\odot \left( \frac{M}{M_\odot} \right)^{4/7} \left( \frac{R}{R_\odot} \right)^2 & = L_\odot \left( \frac{M}{M_\odot} \right)^{11/2} \left( \frac{R}{R_\odot} \right)^{-1/2}\\
0.2 \left( \frac{M}{M_\odot} \right) ^{4/7 - 11/2} & = \left( \frac{R}{R_\odot} \right)^{-1/2 -2}\\
0.2 \left( \frac{M}{M_\odot} \right)^{8/14 - 77/14} & = \left( \frac{R}{R_\odot} \right)^{-5/2}\\
\left( 0.2 \left( \frac{M}{M_\odot} \right)^{-69/14} \right)^{-2/5} & = \frac{R}{R_\odot}
\end{align}

\begin{align}
L_{conv} &  = 0.2L_\odot \left( \frac{M}{M_\odot} \right)^{4/7} \left( \left( 0.2 \left( \frac{M}{M_\odot} \right)^{-69/14} \right)^{-2/5} \right)^2\\
& = 0.2L_\odot \left( \frac{M}{M_\odot} \right) ^{4/7} \left( 0.2 \left( \frac{M}{M_\odot} \right)^{-69/14} \right)^{-4/5}\\
& = 0.2L_\odot \left( \frac{M}{M_\odot} \right)^{4/7} (0.2)^{-4/5} \left( \frac{M}{M_\odot} \right)^{69/7 \cdot 2/5}\\
& = 0.2L_\odot \left( \frac{M}{M_\odot} \right)^{4/7} (0.2)^{-4/5} \left( \frac{M}{M_\odot} \right)^{138/35}\\
& = 0.2^{-1/5}L_\odot \left( \frac{M}{M_\odot} \right)^{158/35}\\
\end{align}
\newpage

Problem 2a:\\
\indent For KH, $t_{KH} > t_{FF}$. The KH timescale is the one where all of the energy of the sun is released through contraction. It's going to be larger than the freefall timescale because there's something opposing the contraction and turning that energy into the luminosity we see. If the KH timescale is less than the freefall timescale, well, that means that there's some force that is either resisting the gravitational collapse or assisting the KH contraction, or a combination of both.
\newpage

Problem 2b:

\begin{align}
t_{KH} & = t_{FF}\\
\frac{E}{L} & = \frac{1}{\sqrt{G \bar{\rho}}}\\
\frac{U}{2L} & = \frac{1}{\sqrt{G \bar{\rho}}}\\
\frac{GM^2}{2RL} & = \frac{1}{\sqrt{G \bar{\rho}}}\\
\frac{GM^2}{2R \cdot 0.2L_\odot \left( \frac{M}{M_\odot} \right)^{4/7} \left( \frac{R}{R_\odot}\right)^2} & = \dfrac{1}{\sqrt{G\dfrac{M}{\dfrac{4}{3} \pi R^3}}}\\
\frac{GM^2}{2R \cdot 0.2L_\odot \left( \frac{M}{M_\odot} \right)^{4/7} \left( \frac{R}{R_\odot}\right)^2} & = \sqrt{\frac{4\pi R^3}{3GM}}\\
\frac{GM^2}{0.4RL_\odot \left( \frac{M}{M_\odot} \right) ^{4/7} \left( \frac{R}{R_\odot} \right)^2} & = \sqrt{\frac{4\pi R^3}{3GM}}\\
\frac{25G^2M^4}{4R^2L_\odot^2 \left( \frac{M}{M_\odot} \right)^{8/7} \left( \frac{R}{R_\odot} \right)^4} & = \frac{4 \pi R^3}{3GM}\\
\frac{75 G^3 M^5}{16 \pi L_\odot^2 \left( \frac{M}{M_\odot} \right)^{8/7}} & = R^5 \left( \frac{R}{R_\odot} \right)^4\\
\frac{75G^3M^5R_\odot^4}{16 \pi L_\odot^2 \left( \frac{M}{M_\odot} \right)^{8/7}} & = R^9\\
\left( \frac{75G^3M^5R_\odot^4}{16 \pi L_\odot^2 \left( \frac{M}{M_\odot} \right)^{8/7}} \right)^{1/9} & = R\\
\left( \frac{75G^3M^5 M_\odot^{8/7}R_\odot^4}{16 \pi L_\odot^2 M^{8/7}} \right)^{1/9} & = R\\
\left( \frac{75G^3M^{27/7} M_\odot^{8/7}R_\odot^4}{16 \pi L_\odot^2} \right)^{1/9} & = R\\
\left( \frac{75G^3}{16 \pi L_\odot^2}  M^{27/7} M_\odot^{8/7} R_\odot^4 \right)^{1/9} & = R
\end{align}

Problem 2c:

\begin{align}
2K &  = U\\
2 \frac{3}{2}NkT & = \frac{GM^2}{R}\\
\frac{3MkT}{\mu m_p} & = \frac{GM^2}{R}\\
T & = \frac{\mu m_p GM}{3kR}~, \text{setting} ~\mu = 1\\
& = \frac{\mu m_p GM}{3k} \left( \frac{75G^3}{16 \pi L_\odot^2}  M^{27/7} M_\odot^{8/7} R_\odot^4 \right)^{-1/9}\\
& = M^{4/7} (5.11 \times 10^{-2} ~K)
\end{align}
\end{document}