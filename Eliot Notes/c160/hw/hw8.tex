\documentclass[10pt,letter,preprint]{aastex}
\usepackage[latin1]{inputenc}
\usepackage{amsmath}
\usepackage{amsfonts}
\usepackage{amssymb}
\usepackage{mathtools}
\usepackage{fullpage}

\newcommand{\pt}{\propto}
\newcommand{\rp}{\right)}
\newcommand{\lp}{\left(}
\newcommand{\half}{\frac{1}{2}}
\newcommand{\mfrac}{\lp \frac{M}{M_\odot}\rp}
\newcommand{\rfrac}{\lp \frac{R}{R_\odot} \rp}
\newcommand{\ra}{\rightarrow}
\newcommand{\la}{\leftarrow}

\begin{document}

\title{HW \#8}
\author{\begin{large}Jeren Suzuki\end{large}}
\author{October 28, 2011}

Problem 1a:
$\rho_{min} = $?, if $T \sim 300$ K?

\begin{align*}
P_{gas} &= P_{degen}\\
\frac{\rho kT}{\mu m_p} &= \frac{h^2}{5m_e}\lp \frac{3}{8 \pi} \rp^{2/3} n^{5/3}\\
\frac{\rho kT}{\mu m_p} &= \frac{h^2}{5m_e}\lp \frac{3}{8 \pi} \rp^{2/3} \lp \frac{\rho}{\mu m_p} \rp^{5/3}\\
\frac{ kT}{\mu m_p} (\mu m_p)^{5/3}\frac{5m_e}{h^2} \lp \frac{8 \pi }{3} \rp^{2/3} &=\frac{\rho^{5/3}}{\rho}\\
\frac{ kT}{\mu m_p} (\mu m_p)^{5/3}\frac{5m_e}{h^2} \lp \frac{8 \pi }{3} \rp^{2/3} &= \rho^{2/3}\\
\lp \frac{ kT}{\mu m_p} (\mu m_p)^{5/3}\frac{5m_e}{h^2} \lp \frac{8 \pi }{3} \rp^{2/3} \rp^{3/2} &= \rho\\
.0146 \text{ g cm}^{-3} &\approx \rho
\end{align}

\begin{align*}
P_{gas} &= P_{degen}\\
\frac{\rho kT}{\mu m_p} &= \frac{h^2}{5m_e}\lp \frac{3}{8 \pi} \rp^{2/3} n^{5/3}\\
\frac{\rho kT}{\mu m_p} &= \frac{h^2}{5m_e}\lp \frac{3}{8 \pi} \rp^{2/3} \lp \frac{\rho}{\mu m_p} \rp^{5/3}\\
T &= \frac{h^2}{5m_e}\lp \frac{3}{8 \pi} \rp^{2/3} \lp \frac{\rho}{\mu m_p} \rp^{5/3} \frac{\mu m_p}{\rho k}\\
T &= \frac{h^2}{5m_e}\lp \frac{3}{8 \pi} \rp^{2/3} \lp \frac{\rho}{\mu m_p} \rp^{2/3} \frac{1}{k}\\
T &\approx 142.57 \text{ K}
\end{align}

\newpage
Problem 1b:

Compare $E_E$, $E_F$, and $E_{thm}$. 

\begin{align*}
E_E &= \frac{(Z_1 Z_2 q_1 q_2)}{r}\\
&= \frac{(Z_1 Z_2 q_1 q_2)}{n^{-1/3}}\\
&=(Z_1 Z_2 q_1 q_2)n^{1/3}\\
&=(Z_1 Z_2 q_1 q_2) \lp \frac{\rho}{\mu m_p} \rp^{1/3}\\
\Aboxed{E_E &\approx 2.61 \times 10^{-10} \text{ ergs}}
\end{align}

\begin{align*}
E_F &= \frac{h^2}{2m} \lp \frac{3}{8 \pi} \rp^{2/3} \lp \frac{\rho}{\mu m_p} \rp^{2/3}\\
&\approx 8.6 \times 10^{-12} \text{ ergs}
\end{align}

\begin{align*}
E_{thm} &= \frac{3}{2}kT\\
E_{thm} &\approx 6.2 \times 10^{-14} \text{ ergs}
\end{align}

$E_E$ is the dominant force. 

Problem 2a:

\begin{align*}
E_G &= \frac{2 \pi^2 m_r e^4 Z_1^2 Z_2^2}{\hbar^2}\\
\Aboxed{&\approx 4.22 \times 10^{-6} \text{ ergs}}
\end{align}

\begin{align*}
\Aboxed{< \sigma v> &= 2.65 S(E_0)\frac{E_G^{1/6}}{(kT)^{2/3} \sqrt{m_r}}e^{-3\sqrt[3]{{(E_G/4kT)}}}}\\
< \sigma v> &= 4.78\times 10^{-15} T^{-2/3} e^{-5909T^{-1/3}}
\end{align}

\newpage
Problem 2b:
%\begin{align*}
%R &= R_\odot \lp \frac{M}{M_\odot} \rp^{1/2} \lp \frac{\text{t}}{2\times 10^7 \text{ years}} \rp^{-1/3}\\
%\frac{dR}{dt} &= - R_\odot \lp \frac{M}{M_\odot} \rp^{1/2} \frac{1}{3}t^{-4/3} ( 2 \times 10^7 \text{ years})^{1/3}\\
%t_c &= \frac{R}{\Bigl\lvert \frac{dR}{dt} \Bigl\lvert}\\
%\Aboxed{t_c &=3 \frac{R}{R_\odot} \lp \frac{M_\odot}{M} \rp^{1/2} t^{4/3}(2 \times 10^7\text{ years})^{-1/3}} 
%\end{align}

As radius decreases, $t_c$ increases because $t \pt R^{-3} \rightarrow t_c \pt R\times R^{-4} = R^{-3}$.

\begin{align*}
L = \frac{GM^2}{2R^2} \Bigl\lvert \frac{dR}{dt} \Bigl\lvert &= 0.2 L_\odot \mfrac^{4/7} \rfrac^2\\
\Bigl\lvert \frac{dR}{dt} \Bigl\lvert &= 0.2 L_\odot \mfrac^{4/7} \rfrac^2 2 \rfrac^2 R_\odot^2 \frac{1}{GM_\odot^2} \mfrac^{-2}\\
\Bigl\lvert \frac{dR}{dt} \Bigl\lvert &= 0.4 L_\odot \mfrac^{-10/7} \rfrac^4 \frac{R_\odot^2}{GM_\odot^2}\\
t_c = \frac{R}{\Bigl\lvert \frac{dR}{dt} \Bigl\lvert} &= \frac{1}{0.4 L_\odot \mfrac^{-10/7} \rfrac^4 \frac{R_\odot^2}{GM_\odot^2}}\\
\Aboxed{&= \mfrac^{10/7} \rfrac^{-3}\frac{5GM_\odot^2}{2R_\odot L_\odot}}
\end{align}


Problem 2c:

\begin{align*}
t_D &= \frac{1}{n_p <\sigma v>}\\
n_p &= \frac{\rho}{\mu m_p}\\
t_D(\rho,T) &= \frac{\mu m_p}{\rho<\sigma v>}~,
\end{align}

where $< \sigma v>$ is defined above.

\begin{align*}
\rho_c &= \bar{\rho}a_n~,a_n=5.99\\
T_c &= 7.5 \times 10^6 \mfrac \rfrac^{-1}\text{ K}~,
\end{align}
 and plug those into $< \sigma v>$ to get it in terms of $M$ and $R$. 
 
\begin{align*}
t_D(\rho,T) &= \frac{\mu m_p}{\rho<\sigma v>}\\
&= \frac{\mu m_p 4 \pi R^3}{3M a_n} \frac{1}{<\sigma v>}\\
<\sigma v> &= 1.2 \times 10^{-19} \mfrac^{-2/3} \rfrac^{2/3} e^{-4.4 \times 10^{10} \mfrac^{-1/3} \rfrac^{1/3}}\\
\frac{1}{<\sigma v>} &= 8.33 \times 10^{18} \mfrac^{2/3} \rfrac^{-2/3} e^{4.4 \times 10^{10} \mfrac^{-1/3} \rfrac^{1/3}}\\
t_D(M,R) &= \frac{\mu m_p 4 \pi}{3a_n} \rfrac^3 \mfrac^{-1} \frac{R_\odot^3}{M_\odot} 8.33 \times 10^{18} \mfrac^{2/3} \rfrac^{-2/3} e^{4.4 \times 10^{10} \mfrac^{-1/3} \rfrac^{1/3}}\\
\Aboxed{t_D(M,R) &=  \frac{\mu m_p 4 \pi}{3a_n} \mfrac^{-1/3} \rfrac^{7/3}  \frac{R_\odot^3}{M_\odot}  8.33 \times 10^{18}  e^{4.4 \times 10^{10} \mfrac^{-1/3} \rfrac^{1/3}}}
\end{align} 
 
 Problem 2d:
%\begin{align*}
%t_D &= t_c\\
%\frac{\mu m_p}{\rho<\sigma v>} &= 3 \frac{R}{R_\odot} \lp \frac{M_\odot}{M} \rp^{1/2} t^{4/3}(2 \times 10^7\text{ years})^{-1/3}\\
%\frac{\mu m_p R^3 4 \pi}{3 a_n M<\sigma v>}&= 3 \frac{R}{R_\odot} \lp \frac{M_\odot}{M} \rp^{1/2} t^{4/3}(2 \times 10^7\text{ years})^{-1/3}\\
%\frac{\mu m_p 4 \pi}{3 a_n <\sigma v>} \frac{R^3}{M}&= 3 \frac{R}{R_\odot} \lp \frac{M_\odot}{M} \rp^{1/2} t^{4/3}(2 \times 10^7\text{ years})^{-1/3}\\
%\frac{\mu m_p 4 \pi}{3 a_n <\sigma v>} \rfrac^3 R_\odot^3 \frac{M_\odot}{M} \frac{1}{M_\odot} &= 3 \frac{R}{R_\odot} \lp \frac{M_\odot}{M} \rp^{1/2} t^{4/3}(2 \times 10^7\text{ years})^{-1/3}\\
%\rfrac^2 &= \frac{3 a_n <\sigma v >}{4 \pi \mu m_p} 3 \mfrac^{-1/2} \frac{M_\odot}{R_\odot^3}  t^{4/3} (2 \times 10^7 \text{ years} )^{-1/3}\\
%\frac{R}{R_\odot} &= \lp \frac{3 a_n <\sigma v >}{4 \pi \mu m_p} 3 \mfrac^{-1/2} \frac{M_\odot}{R_\odot^3}  t^{4/3} (2 \times 10^7 \text{ years} )^{-1/3} \rp^{1/2}
% \end{align}

\begin{align*}
t_D(M,R) &=  t_c\\
\frac{\mu m_p 4 \pi}{3a_n} \mfrac^{-1/3} \rfrac^{7/3}  \frac{R_\odot^3}{M_\odot}  8.33 \times 10^{18}  e^{4.4 \times 10^{10} \mfrac^{-1/3} \rfrac^{1/3}} &= \mfrac^{10/7} \rfrac^{-3}\frac{5GM_\odot^2}{2R_\odot L_\odot}\\
\frac{\mu m_p 4 \pi}{3a_n} \rfrac^{16/3}  \frac{R_\odot^3}{M_\odot}  8.33 \times 10^{18}  e^{4.4 \times 10^{10} \mfrac^{-1/3} \rfrac^{1/3}} &= \mfrac^{10/7 + 1/3} \frac{5GM_\odot^2}{2R_\odot L_\odot}\\
\rfrac^{16/3}  e^{4.4 \times 10^{10} \mfrac^{-1/3} \rfrac^{1/3}}&= 1.2 \times 10^{-19} \mfrac^{37/21} \frac{15a_nGM_\odot^3}{8 \mu m_p \pi R_\odot^4 L_\odot}\\
\Aboxed{\rfrac^{16/3}  e^{4.4 \times 10^{10} \mfrac^{-1/3} \rfrac^{1/3}}&= 2.25 \times 10^{21} \mfrac^{37/21}}
\end{align}

\begin{align*}
\rfrac^{16/3}  e^{4.4 \times 10^{10} \mfrac^{-1/3} \rfrac^{1/3}}&= 2.25 \times 10^{21} \mfrac^{37/21}\\
\text{For } M = .03 M_\odot, \boxed{\rfrac=.685}\\
\text{For } M = .1 M_\odot, \boxed{ \rfrac=1.82}\\
\end{align}

\begin{align*}
T_c &= 7.5 \times 10^6 \mfrac \rfrac^{-1}\text{ K}\\
T_c(M=.03M_\odot)&= 7.5 \times 10^6 \times .03 \times .685^{-1}\text{ K}\\
\Aboxed{&= 3.28 \times 10^5 \text{ K}}\\
T_c(M=.1M_\odot)&= 7.5 \times 10^6 \times .03 \times 1.82^{-1}\text{ K}\\
\Aboxed{&= 4.12 \times 10^5 \text{ K}}\\
\end{align}

\begin{align*}
t_D(M=.03M_\odot) = 5.34 \times 10^{13} \text{ s}\\
t_D(M=.1M_\odot) = 1.56 \times 10^{13} \text{ s}
\end{align}

Deuterium fusing happens before the MS because there is some radius $R_D$ where $t_c = t_D$ since we defined $t_D$ as the lifetime of a Deuterium nucleus. As a result, there exists some time where in contraction, Deuterium fuses, which since it's contracting, is therefore before the main sequence.

Problem 2e:

\begin{align*}
L =- \frac{1}{2}\frac{GM^2}{R}\Bigl\lvert \frac{dR}{dt} \Bigl\lvert &= 5.5\text{ MeV} \times \frac{\text{reactions}}{\text{sec}}\\
-\frac{1}{2}\frac{GM^2}{t_c} &= 5.5\text{ MeV} \times \frac{\text{reactions}}{\text{sec}}\\
-\frac{1}{2}\frac{GM^2}{t_c} &= 8.8 \times 10^{-6} \text{ ergs}  \times \frac{\text{reactions}}{\text{sec}}\\
10^{34} \text{ ergs s}^{-1} &\simeq  8.8 \times 10^{-6} \text{ ergs}  \times n_1n_2<\sigma v>\\
10^{34} \text{ ergs s}^{-1} &\simeq 10^{-21} \text{ ergs s}^{-1} 
\end{align}

We see that the luminosity of Deuterium is much smaller than the fusion luminosity and it will never halt contraction of the star.

\newpage
Problem 3a:

\begin{align*}
\rho_c = \bar{\rho} a_n\\
P_c = GM^{2/3} \rho^{4/3}d_n
\end{align}

\begin{align*}
P_c &= K\rho_c^{5/3}\\
GM^{2/3}\rho_c^{4/3}d_n &= K(\bar{\rho}a_n)^{5/3}\\
GM^{2/3} \lp \frac{3M}{4\pi R^3}a_n \rp^{4/3} d_n &= K \lp \frac{3M}{4\pi R^3} a_n\rp^{5/3}\\
R &= \lp \frac{3Ma_n}{4 \pi} \rp^{5/3} \frac{K(4\pi)^{4/3}}{G(3a_n)^{4/3}M^2d_n}\\
R(M) &= M^{-1/3} \lp \frac{3a_n}{4 \pi} \rp^{5/3} \frac{K(4\pi)^{4/3}}{G(3a_n)^{4/3}d_n}
\end{align}

To find $K$,
\begin{align*}
\frac{h^2}{5m_e}\lp \frac{3}{8\pi}\rp^{2/3}n^{5/3} &= K\rho^{5/3}\\
\frac{h^2}{5m_e}\lp \frac{3}{8\pi}\rp^{2/3}\lp \frac{\rho}{\mu_e m_p} \rp^{5/3} &= K\rho^{5/3}\\
\frac{h^2}{5m_e}\lp \frac{3}{8\pi}\rp^{2/3}\lp \frac{1}{\mu_e m_p} \rp^{5/3}&= K
\end{align}

\begin{align*}
R(M) &= M^{-1/3} \lp \frac{3a_n}{4 \pi} \rp^{5/3} \frac{(4\pi)^{4/3}}{G(3a_n)^{4/3}d_n}\frac{h^2}{5m_e}\lp \frac{3}{8\pi}\rp^{2/3}\lp \frac{1}{\mu_e m_p} \rp^{5/3}\\
\rfrac &= \frac{1}{R_\odot} \mfrac^{-1/3} M_\odot^{-1/3}\lp \frac{3a_n}{4 \pi} \rp^{5/3} \frac{(4\pi)^{4/3}}{G(3a_n)^{4/3}d_n}\frac{h^2}{5m_e}\lp \frac{3}{8\pi}\rp^{2/3}\lp \frac{1}{\mu_e m_p} \rp^{5/3}\\
\Aboxed{\rfrac &= \frac{1}{R_\odot} \mfrac^{-1/3} M_\odot^{-1/3}\lp \frac{3a_n}{4 \pi} \rp^{1/3} \frac{1}{Gd_n}\frac{h^2}{5m_e}\lp \frac{3}{8\pi}\rp^{2/3}\lp \frac{1}{\mu_e m_p} \rp^{5/3}}\\
\end{align}

Problem 3b:

\begin{align*}
\rfrac \text{ where $M = M_J = 1.9 \times 10^{30}$ g},\\
\Aboxed{\rfrac &\approx 0.04}~,\text{ in reality, $\frac{R_J}{R_\odot} \approx 0.1$}
\end{align}

Problem 3c:

\begin{align*}
E_E &= E_F\\
\frac{Z_1Z_2q_1q_2}{r} &\sim \frac{1}{2m} \lp \frac{3h^3}{8 \pi} \rp^{2/3} n^{2/3}~,r \sim n^{-1/3}\\
Z_1Z_2q_1q_2n^{1/3} &\sim \frac{1}{2m} \lp \frac{3h^3}{8 \pi} \rp^{2/3} n^{2/3}\\
Z_1Z_2q_1q_2\lp \frac{\rho}{\mu m_p} \rp^{1/3} &\sim  \frac{1}{2m} \lp \frac{3h^3}{8 \pi} \rp^{2/3} \lp \frac{\rho}{\mu m_p} \rp^{2/3}\\
\lp \lp \frac{8 \pi}{3h^3} \rp^2 2m_e Z_1Z_2q_1q_2 \rp^3 \mu m_p &\sim \rho\\
\boxed{0.0984 \text{ g cm}^{-3}\sim \rho}
\end{align}

\begin{align*}
.1 &= \rho_c &= \frac{3Ma_n}{4\pi R^3}\\
\frac{.4 \pi R^3}{3a_n} &= M\\
\frac{.4 \pi}{3a_n} R^3 &= M\\
\frac{.4 \pi R_\odot^3}{3a_n M_\odot} \rfrac^3 &= \mfrac\\
\frac{.4 \pi R_\odot^3}{3a_n M_\odot} \lp \frac{1}{R_\odot} \mfrac^{-1/3} M_\odot^{-1/3}\lp \frac{3a_n}{4 \pi} \rp^{1/3} \frac{1}{Gd_n}\frac{h^2}{5m_e}\lp \frac{3}{8\pi}\rp^{2/3}\lp \frac{1}{\mu_e m_p} \rp^{5/3}  \rp^3 &= \mfrac\\
\frac{.4 \pi R_\odot^3}{3a_n M_\odot} \lp \frac{1}{R_\odot}  M_\odot^{-1/3}\lp \frac{3a_n}{4 \pi} \rp^{1/3} \frac{1}{Gd_n}\frac{h^2}{5m_e}\lp \frac{3}{8\pi}\rp^{2/3}\lp \frac{1}{\mu_e m_p} \rp^{5/3}  \rp^3 &= \mfrac^2\\
\frac{.4 \pi R_\odot^3}{3a_n M_\odot} \lp \frac{1}{R_\odot}  M_\odot^{-1/3}\lp \frac{3a_n}{4 \pi} \rp^{1/3} \frac{1}{Gd_n}\frac{h^2}{5m_e}\lp \frac{3}{8\pi}\rp^{2/3}\lp \frac{1}{\mu_e m_p} \rp^{5/3}  \rp^{3/2} &= \mfrac\\
9.6 \times 10^{-5} &=\mfrac\\
\Aboxed{1.92 \times 10^{29} \text{ g} &= M}
\end{align}

\begin{align*}
R &= \lp \frac{3Ma_n}{.4 \pi} \rp^{1/3}\\
&= 1.4 \times 10^{10} \text{ cm}\\
\Aboxed{&\approx .201 R_\odot}
\end{align}

\begin{align*}
E_F &> E_C~,\text{ then degenerate}\\
n^{2/3} &\gtrsim n^{1/3}
\end{align}

Since $n_Q$ is proportional to T, and since $E_F \pt n^{2/3}$, we are at high high $n$ for the degenerate object that we don't care about the $T$ dependence. $E_F$ will be significantly larger than $E_C$ which is in turn larger than $E_{thm}$. If you get to a regime that $E_F \sim E_C$, then that will be the maximum radius since you can get smaller radii and still be degenerate.

\end{document}