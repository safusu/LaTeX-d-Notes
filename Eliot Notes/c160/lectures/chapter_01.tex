\chapter{Introduction}

\begin{center}
\textbf{\begin{huge} October 25, 2011\end{huge}}
\end{center}

\section{Intro to cgs}

Mechanics:
\begin{center}
\begin{tabular}{c|c}
\hline
cgs &SI \\ \hline
cm & m\\ \hline
g & kg\\ \hline
s & s\\ \hline
erg & J\\ \hline
ergs/s & W\\ \hline
dyne & N \\
\hline
\end{tabular}
\end{center}

E\&M
\begin{center}
\begin{tabular}{c|c}
\hline
cgs &SI \\ \hline
k=1 & k=$\frac{1}{4\pi \epsilon_0}$ \\ \hline
$q_e = 4.8 \times 10^{-10}$ esu & $q_e = C$\\ \hline
$\mathbf{F} = q \frac{\mathbf{v}}{c} \times \mathbf{B} + q\mathbf{E}$ & $\mathbf{v}\times\mathbf{B}$\\  % WAIT WHAT
\hline
\end{tabular}
\end{center}

\section{Intro to Stars}

Sun:
\begin{list}{$^\circ$}{}
\item $L_\odot = 3.8 \times 10^{33} \textrm{ ergs s}^{-1} = 3.8 \times 10^{26}\textrm{ W}$
\item $R_\odot = 7 \times 10^{10} \textrm{ cm} \sim 10R_{\textrm{Jupiter}}$
\item $M_\odot = 2 \times 10^{33} \textrm{ g} \sim 10^3 M_{\textrm{Jupiter}} \sim 3\times 10^5 M_\oplus$
\item $t_\odot = 4.5 \times 10^9 \textrm{ years}$ 
\item $F_{BB} = \sigma T_{eff}^4$
\item $L = 4 \pi R_\odot^2 \sigma T_{eff}^4~, T_{eff} \approx 5800\textrm{ K}$
\item $T_c = 1.56 \times 10^7$ K, from neutrino measurements and sound waves
\end{list}

\begin{align}
< \rho_\odot> &= \frac{M}{\frac{4}{3}\pi R^3} \approx 1\textrm{ g cm}^{-3}\sim \rho_{\textrm{water}}\\
\rho_c &= 150 \textrm{ g cm}^{-3}\\
n &= \frac{\rho}{\mu m_p} \approx 10^{26} \textrm{ cm}^{-3}\\
&= n_e = n_p
\end{align}

Another way, $d$ = distance between particles; i.e., one particle per sphere of radius $d$. 
\begin{align}
n = \frac{4}{3}\pi d^3 = 1\\
d \approx \lp \frac{n}{\frac{4}{3}\pi }\rp^{-1/3} \sim n^{-1/3}
\end{align}
At the center of the sun, $d \approx 2 \times 10^{-9}$ cm, which is about the B\"ohr Radius ($a_0 \sim 10^{-8}$ cm).

An $e^-$ has 0 volume whereas if the size of a nucleus is $\sim 10^{-13}$ cm $ \ll$ distance between particles, then we can neglect inter-molecular forces and approximate the gas as an ideal gas. 

\section{Ideal Gas}

\begin{align}
P = n_e kT + n_p kT = 2n_p kT
\end{align}
At the center, $P \approx 2 \times 10^{17} \textrm{ ergs cm}^{-3}$. Photons have an intrinsic radiation pressure given by:
\begin{align}
P_{rad} &= \frac{1}{3}aT^4 \approx 10^{14}\textrm{ ergs cm}^{-3}~,a \textrm{ is the radiation constant}
\end{align}
If photons were degenerate, then instead:
\begin{align}
P = \frac{h^2}{5m_e} \lp \frac{3}{8\pi}^{2/3} \rp n_e^{5/3},
\end{align}
which stands true for an ideal QM gas of fermions, giving us $P_{\textrm{degen}} \approx 5 \times 10^{16} \textrm{ ergs cm}^{-3}$ at the center of the sun. $P_{\textrm{ideal}} \sim P_{\textrm{degen}}$, so our classical assumption of the state of gas in the sun isn't great. 

\subsection{Other Stars}

\begin{list}{$^\circ$}{}
\item$M \sim 0.1M_\odot - 100M_\odot$
\item $R \sim 0.1 - 1000R_\odot$
\item $L \sim 10^{-4} - 10^6 L_\odot$
\item $T_{eff} \sim \textrm{ few } 10^3 - 5 \times 10^4 \textrm{ K}$
\end{list}

On the Main Sequence, plotting $\ln \lp \frac{L}{L_\odot} \rp$ as a function of  $\ln \lp \frac{M}{M_\odot} \rp$, we see that $R \sim M^{0.75}, L \sim M^{3.5}, L\sim T^6$. For the most part, while on the MS, a star fuses H into He. Most stars spend most of their time in this phase. 
