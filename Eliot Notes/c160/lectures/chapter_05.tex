\chapter{Conduction}

\begin{center}
\textbf{\begin{huge} September 8, 2011\end{huge}}
\end{center}

\section{Picking Up Where We Left Off}
\begin{align}
F &= -\frac{4}{3}\frac{caT^3}{\kappa \rho} \frac{dT}{dr} \leftarrow \text{ for radiative diffusion}\\
&= -\frac{1}{3}lc \frac{dU}{dr}
\end{align}

Applying it:

\begin{align}
\frac{dT}{dr} \sim -\frac{T}{R}\\
L \sim \chi M^3~, kT \sim \frac{GM \mu m_p}{R}\\
\chi = \frac{a (\mu m_p)^4 cG^4 \mu_e m_p}{\sigma_T \kappa^4}\\
L \sim 10^{35} \lp \frac{M}{M_\odot} \rp ^3 \text{ ergs/s}\\
L \sim L_\odot~,e^- \text{ conduction $L \ll L_\odot$ so photons are dominating}
\end{align}

$L$ is strictly \textit{NOT} dependent on fusion, it is more dependent on $\frac{dU}{dT}$ of the photons. Fusion creates $E$ and the photons but doesn't determine the rate of energy leaving.\\
$\chi$ is constant, but dependent on the composition of the star. 

\begin{align}
L \pt \mu^4 \mu_e M^3\\
H \rightarrow HE\\
\mu \rightarrow \mu \uparrow\\
L \rightarrow L \uparrow
\end{align}

$L$ was lower in the past. $L$ when the Earth forced was only 20\% of the current $L_\odot$. This brings about the problem that the Earth was supposedly a ball of ice during it's evolution. \\

\begin{align}
\frac{3}{2}nkT > aT^4\\
F &= -\frac{1}{3}lv \frac{dU}{dx}\\
&= -\chi \frac{dT}{dx},\chi = \frac{1}{3}lv \frac{dU}{dT}\\
\frac{F_{rad}}{F_{e^-}} &= \frac{-\chi_{rad}}{-\chi_{e^-}}, \frac{dT}{dx}\text{ is the same for both}\\
&\sim \frac{l_\gamma}{l_{e^-} }\frac{ c}{v_{e^-} }\frac{aT^4}{\frac{3}{2}nkT}
\end{align}

$ \frac{l_\gamma}{l_{e^-} } \ggg 1$, $\frac{ c}{v_{e^-}} \gg 1$, and $\frac{aT^4}{\frac{3}{2}nkT} \ll 1$.\\

\begin{align}
F = -\frac{4}{3}\frac{caT^4}{n \sigma} \frac{dT}{dr}
\end{align}

We want to find the time for thermal energy to leak out by photon diffusion.

\begin{align}
t_{KH} &= \frac{E}{L}\\
L &\sim 4 \pi R^2 \frac{4}{3}aT^3 \frac{1}{n\sigma}\frac{T}{R}\\
&= \frac{\frac{3}{2}nkT \cdot \dfrac{4}{3}\pi R^3}{ \lp \dfrac{4 \cdot 16 \pi aT^4R}{R3n\sigma} \rp}\\
&=\frac{\frac{3}{2}nkTR^2n\sigma}{4aT^4c}\\
&\sim \frac{nkT}{aT^4}\frac{R^2}{lc}\\
&\sim \frac{nkT}{aT^4}t_{RW}~,\text{ where $t_{RW}=\frac{R^2}{lc}=$ random walk time}
\end{align}

\begin{align}
<|D|^2>&=Nl^2~,\text{ where $N = $ number of steps}\\
<|D|^2>^{1/2} &= \text{ RMS Distance = typical distance a photon will find itself after $N$ scatterings}\\
&= \sqrt{N}l
\end{align}

What is $N$ so that the photon leaves the star?

\begin{align}
<|D|^2>^{1/2} &=R\\
N &\sim \lp \frac{R}{l} \rp^2\\
\frac{R}{l} \sim 10^{11},N\sim 10^{22}
\end{align}

How long does it take to get out? 

\begin{align}
Nt_{step} = t_{esc} = N\frac{l}{c} \sim \lp \frac{R}{l} \rp^2 \cdot \frac{l}{c} \sim \frac{R^2}{lc}\\
\sim 10^4 \text{ years for our sun}
\end{align}

If photons didn't bounce around, it would escape in 2 seconds. ($\nu_e$ can get out in about 2 seconds, $l_\nu$ must be greater than $R_\odot$) Also, the time it takes for heat to diffuse throughout a room is: $\frac{R^2}{lv_{thm}}$.

\begin{align}
F &= -\frac{4}{3}\frac{caT^4}{n \sigma} \frac{dT}{dr}\\
&= -lc \frac{d}{dr}\frac{1}{3}aT^4\\
F_r &= -lc\frac{d}{dr}P_{rad}~,\text{ we must be careful, $F$ and $L$ depend on $r$}\\
\frac{L_r}{4\pi r^2} &= -lc \frac{d}{dr}P_{rad}\\
-\frac{L_r}{4\pi lc r^2} &= \frac{d}{dr}P_{rad}\\
-\frac{L_r \kappa \rho}{4\pi c r^2} &= \frac{d}{dr}P_{rad}\\
\end{align}

We're interested in how $P_{rad}$ changes with $P_{tot}$. 

\begin{align}
\frac{dP_{rad}}{dP} = \frac{L_r \kappa}{4 \pi G M_r c} \equiv \frac{L_r}{L_{EDD}}
\end{align}

Roughly, if $P_{rad} \sim P_{tot}$, then $L_r \sim L_{EDD}$.

\subsection{Eddington Luminosity}

\begin{list}{}{}
\item $F_g = -\frac{GMm}{r^2}$
\item $F_{rad} = \frac{dp}{dt}$
\item $p_{proton} = \frac{E}{c} = \frac{h \nu}{c} = \frac{h}{\lambda}$
\end{list}

What is the total $p$ per unit time produced by the star? $\sum\limits{_i^\infty} p_i $ is too hard...

\begin{align}
p_{photon} = \frac{E}{c} = \frac{L}{c}\\
F_{Rad} = \frac{dp}{dt} = \frac{L}{c 4\pi r^2} \sigma
\end{align}

The Eddington Luminosity is where the radiation force equals the forge of gravity. The Eddington "Limit" is:

\begin{align}
L_{EDD} = \frac{4 \pi G Mc}{\sigma /m} = \frac{4 \pi G Mc}{\kappa}
\end{align}

If $L > L_{EDD}$, $F_{rad} > F_{grav}$, and material is "blown" out. 

\subsection{Fully Ionized H}

\begin{list}{$\circ$}{}
\item $L_{EDD} = \frac{4 \pi GMc}{\kappa}$
\item $\kappa = \frac{\sigma_T}{m}$
\item $l = \frac{1}{n \sigma} = \frac{1}{n_e \sigma_T} = \frac{1}{\kappa \sigma}$
\item For fully ionized $H$, $\mu_e =1$
\item since photons are only interacting with $e^-$ and not $p$, $\mu = 1 = \mu_e$
\end{list}

\begin{align}
\kappa &= \frac{\sigma_T}{m}\\
&= 0.4\text{ cm}^2\text{ g}^{-1}
\end{align}

\begin{align}
L_{EDD} = 1.3 \times 10^{38} \lp \frac{M}{M_\odot} \rp \text{ ergs s}^{-1}
\end{align}

\begin{list}{}{}
\item If $L \sim L_{EDD}$, $F_{rad} \sim F_{grav}$ which results in the radiation force not being important in the sun. It's dominated by gas pressure.
\item As $M \uparrow, L_{EDD} \uparrow$ so higher $M$ means $P_{rad}$ becomes more important $\rightarrow$ it becomes the dominant force
\end{list}
