
\chapter{Finishing Convection}

\begin{center}
\textbf{\begin{huge} September 20, 2011\end{huge}}
\end{center}

\section{Convection Continued}

\begin{align}
a = |N|^2 \delta r\\
|N|^2 &= \frac{g}{c_p} \Bigl\lvert \frac{ds}{dr} \Bigl\lvert\\
&= \frac{g}{H} \Bigl\lvert \frac{H}{c_p} \frac{ds}{dr} \Bigl\lvert\\
v_c^2 = a \delta r = |N|^2 \delta r^2\\
\delta r \equiv \alpha H\\
v_c = \alpha c_s \Bigl\lvert \frac{H}{c_p} \frac{ds}{dr} \Bigl\lvert^{1/2}\\
F = \frac{1}{2}\rho v_c^3 = \frac{1}{2} \rho \alpha ^3 c_s^3 \Bigl\lvert \frac{H}{c_p} \frac{ds}{dr} \Bigl\lvert ^{3/2}\\
c_s^2 = \frac{kT}{\mu m_p} = \frac{P}{\rho}
\end{align}

We wanted to find the $F_r = -\frac{4}{3} \frac{caT^3}{\kappa \rho}\frac{dT}{dr}$ equivalent for convection. $F = \frac{1}{2}\rho v_c^3 = \frac{1}{2} \rho \alpha ^3 c_s^3 \Bigl\lvert \frac{H}{c_p} \frac{ds}{dr} \Bigl\lvert ^{3/2}$ gives the $v_c$ and $\frac{ds}{dr}$ given the flux. \\

$\Bigl\lvert \frac{H}{c_p} \frac{ds}{dr} \Bigl\lvert \sim 10^{-6}$, $s \sim c_p$, so $\frac{\Delta s }{s} \sim 10^{-6}$ on a length scale $\sim H$. Ergo, $s$ = constant in the convection zone. This replaces $F_r = -\frac{4}{3} \frac{caT^3}{\kappa \rho}\frac{dT}{dr}$. Let's assume $P \pt \rho^\gamma$ \& $\frac{dP}{dr} = -\rho \frac{GM_r}{r^2}$. 

\begin{align}
\frac{dM_r}{dr} = 4 \pi r^2 \rho\\
\frac{d}{dr} \lp \frac{r^2}{\rho} \frac{dP}{dr} = -\rho GM_r \rp\\
\frac{d}{dr} \lp \frac{r^2}{\rho	} \frac{dP}{dr} \rp = -4\pi r^2 G\rho
\end{align}

But... if $P = K \rho^\gamma$, $\frac{dP}{dr} = \gamma K \rho^{\gamma-1} \frac{d\rho}{dr}$! These kinds of models are called:

\subsection{Polytropic Models}

\begin{align}
P &= K\rho^\gamma\\
&= K \rho^{1 + 1/n}, \gamma = 1 + \frac{1}{n}, \text{ where n is the polytropic index}
\end{align}

\begin{align}
\theta = \lp \frac{\rho}{\rho_c} \rp ^{1/n}~,\rho_c = \rho(r=0)\\
\zeta = \frac{r}{a}~, a = \sqrt{\frac{(n+1) K \rho_c ^{\frac{1}{n} -1}}{4 \pi G}}~,[a] = \text{ cm}
\end{align}

\begin{align}
\underbrace{\boxed{\frac{1}{\zeta^2} \cdot \frac{d}{d\zeta} \lp  \zeta^2 \frac{d\theta}{d\zeta} \rp = -\theta^n}}_{\text{Lane-Emden Equation}}
\end{align}

Boundary conditions:

\begin{align}
\theta (0) = 1\\
\frac{d \theta}{d \zeta} \Bigl\lvert_{r=0}\\
\frac{d \theta}{d \zeta} \pt \frac{d \rho}{dr} \pt \frac{dP}{dr} \sim -\frac{\rho GM_r}{r^2} \pt r \text{ as } r \rightarrow 0\\
\frac{dM_r}{dr} = 4 \pi r^2 \rho\\
M_r \sim \rho r^3
\end{align}

Let's look at the properties of a fully convective star of low mass. Low mass $\rightarrow$ low $T$ $\rightarrow$ high $\kappa$. 

\subsection{$M_\star < \frac{1}{3} M_\odot$ on Main Sequence - Pre and Post-Stars Too (Giants)}

For stars with photons carrying the energy out, $L \pt M^3$ if $\sigma = \sigma_T$. For fully convective stars, $L = 4 \pi R^2 F_c$, where $F_c = \rho v_c^3 \pt \Bigl\lvert \frac{ds}{dr} \Bigl\lvert^{3/2}$. Let's look at the surface where photons are carrying the energy out. 

\begin{align}
\text{s = constant}\\
P \pt \rho^{5/3} \pt T^{5/2}\\
\rho T \pt \rho^{5/3}, T \pt \rho^{2/3}\\
\frac{P_c}{P_{photons}} = \lp \frac{T_c}{T_{eff}} \rp ^{5/2},\text{ now use V.T. to relate $T_c$ and $M$ \& $R$.}
\end{align}
\begin{center}
{\Huge BUT}
\end{center}
We know that this is a $n=\sfrac{3}{2}$ polytrope so $kT_c = .54 \frac{GM\mu m_p}{R}$

\begin{align}
P_c = 0.77 \frac{GM^2}{R^4}\\
\frac{dP}{dr} = -\rho \frac{GM_r}{r^2}\\
\frac{P_c}{R} \sim \frac{M}{R^3}\frac{GM}{R^2}\\
P_c \sim \frac{GM^2}{R^4}
\end{align}

\begin{align}
\frac{1}{\kappa_{ph}\rho_{ph}} &= \frac{kT_{eff}}{mg} \\
&= \frac{kT_{eff}R^2}{GMm}\\
\frac{\rho_{ph}kT_{eff}}{m} &= \frac{g}{\kappa_{ph}}\\
&\equiv P_{ph}\\
& = \frac{GM}{R^2\kappa_{ph}}
\end{align}

Now all that's left is $\kappa$. $\kappa_{ph}(\rho_{ph},T_{ph})~, \kappa_{ph} \sim 3 \times 10^{-31} \rho_{ph}^{1/2} T_{eff}^9$. Now we have $\frac{\rho_{ph} kT_{eff}}{m} = \frac{g}{\kappa_{ph}}$. In this example, $\kappa = \kappa_{H^-}$ and $T_{eff} \sim 3000 \sim 10^4$ K. Now we can solve:

\begin{align}
\frac{P_c}{P_{ph}}&= \lp \frac{T_c}{T_{eff}} \rp^{5/2}\\
T_{eff} &= T_c \lp \frac{P_{ph}}{P_c	} \rp ^{2/5}\rightarrow T_{eff}(M,R)~,
\end{align}

and from here you can solve for $T_c(M,R), P_c(M,R), P_{ph}(T_{eff},M,R)$.\\

We're going to assume that $s=$ constant and some other stuff about polytropes. $l \sim H \rightarrow P_{ph}(M,R,T_{eff})$ to get:

\begin{align}
\Aboxed{T_{eff} &\approx 3000 \text{ K }\lp \frac{M}{M_\odot} \rp^{1/7} \lp \frac{R}{R_\odot} \rp^{1/49}}\\
\Aboxed{L = 4 \pi R^2 \sigma T_{eff}^4 &\sim 0.1 L_\odot \lp \frac{M}{M_\odot} \rp^{4/7} \lp \frac{R}{R_\odot} \rp^{102/49}}
\end{align}
for a fully convective star; this is analogous to the $L \pt M^3$ for a radiative star
\begin{align}
T_{eff} &\approx 3000 \text{ K } \lp \frac{L}{L_\odot} \rp^{1/102} \lp \frac{R}{R_\odot} \rp^{7/51}
\end{align}

But the dependence on $M$ and $R$ are so low that fully convective stars share the same $T_{eff} \sim 3000-4000$ K and are pretty much located on the Hayashi Line.

