
\chapter{Post Fusion}
 
\begin{center}
\textbf{\begin{huge} September 29, 2011\end{huge}}
\end{center}

\section{Post-Fusion}

\begin{align}
\lambda \sim \frac{h}{p} \gg & r_n \approx 10^{-13} ~\text{cm}\\
& r_c ~,~\text{classical distance distance, of closest approach}
\end{align}
Fusion parameters will be QM, but won't have stuff to do with $\lambda$. $\sigma$ for nuclear reactions $\approx$ nuclear physics part (strong/weak interaction) $\times$ tunneling through coulomb barrier. We're going to focus on the tunneling which sets the physics for the central temps of stars. \\

Let's consider Schr\"odinger's Equation:

\begin{align}
\left(\frac{\hbar^2}{2m_r} \nabla^2 + V(r) \right) \Psi = E\Psi~,m_r = \text{reduced mass}\\
-\frac{\hbar^2}{2m_r}\frac{d^2}{dr^2} \Psi =E\Psi~, \Psi = e^{ikr}~, E=\frac{\hbar^2 k^2}{2m_r}~,k=\frac{\sqrt{E2m_r}}{\hbar}
\end{align}
Good ol review. $P = | \Psi|^2 = \text{constant}$. Now we have

\begin{align}
\frac{\hbar^2}{2m_r}\frac{d^2}{dr^2} \Psi & = (V-E)\Psi~,\text{and} (V-E)>0\\
\Psi & \propto e^{-kr}~, \frac{\hbar^2 k^2}{2m_r} = (V-E)
\end{align}

Quantum mechanically, particles can't be somewhere where it's potential is less than the energy. Now, the probability of tunneling is $|\Psi|^2 \sim e^{-2kl}$. Tunneling is generic feature of wave theory not just QM. Sound waves tunnel, waves in the atmosphere tunnel...\\
Now lets imagine a particle with energy $E = \frac{1}{2}m_r v^2$,$v = | \overline{v_1} - \overline{v_2} |$. Now...

\begin{align}
\left( -\frac{\hbar^2}{2m_r}\nabla^2 + \frac{Z_1 Z_2 e^2}{r} \right) \Psi = E \Psi\\
E&=V\\
& = \frac{Z_1 Z_2 e^2}{r_c}\\
r_c &= \frac{Z_1 Z_2 e^2}{E}
\end{align}
There's some finite prob that they can tunnel trough the potential one they reach $r_c$. We want to compute the probability! One small difference is that with the atom, particles can have angular momentum and we have to use spherical harmonics.

\begin{align}
\Psi = \frac{f(r)}{r} Y_{l,m}(\theta,\phi)
\end{align}

\begin{align}
\left( -\frac{\hbar^2}{2m_r} \frac{d^2}{dr^2} + \frac{l(l+1)\hbar^2}{2m_r r^2} + \frac{e^2 Z_1 Z_2}{r} \right) \Psi = E \Psi
\end{align}

Particles with high angular momentum don't fuse because any small difference in path and they'll fly off. Only possible fusion happens when $l=0$, so we can cross out the $\frac{l(l+1)\hbar^2}{2m_r r^2} $ term. Now... 

\begin{align}
\left( -\frac{\hbar^2}{2m_r} \frac{d^2}{dr^2} +  \frac{e^2 Z_1 Z_2}{r} \right) f = Ef
\end{align}

\textbf{What's the probability of reaching $r_n$ if they start at the classical turning point ($r_c$)?} $P = | f(r_n)|^2$. 

Now, 

\begin{align}
\frac{d^2 f(r)}{dr^2} + g(r)f(r) = 0~,g(r) = \frac{2m_r}{\hbar^2} \left( E - \frac{e^2 Z_1 Z_2}{r} \right)
\end{align}

We're interested in situations where the $E$ is less than the potential, so $g(r) < 0$. This pops up in  lot of places, apparently. If $g(r)$ is a constant, we can solve it. 

\begin{align}
f \sim e^{\pm i \sqrt{g} r}
\end{align}

This solution isn't valid if $g(r)$ isn't a constant. For the case of interest, $g$ is \textit{almost} constant. It's slowly varying, in reality. It's a function of position for which there is an analytic solution to the above equation. The analytic equation is known as the WKB solution.\\

It's plausible that the solution is of the form $f \sim e^{i \phi(r)}$ if we think $g$ doesn't change much over time. If $g$ = constant, $\phi(r) = \sqrt{g} r$. 

\begin{align}
f' & = i \phi'(r)e^{i \phi(r)} = i \phi'(r)f\\
f'' & =i\phi''f + i \phi'f' = i \phi''f - (\phi')^2f
\end{align}

\begin{align}
\frac{d^2 f(r)}{dr^2} + g(r)f(r) & = 0\\
i\phi'' - (\phi')^2 + g&=0
\end{align}
Assume $\phi''$ is small, and by small we mean $\phi'' \ll g$. 

\begin{align}
(\phi')^2 &= g(r)\\
\phi' & = \sqrt{g(r)}\\
\phi(r) & = \int^r \sqrt{g(x)}dx\\
f & \sim e^{i\phi(r)} = e^{ \pm i\int^r \sqrt{g}dx}
\end{align}

We can check whether our assumption that $\phi'' \ll g$, $\phi'' = \frac{1}{2}g^{-1/2}g'$, and WKB solution is valid if $\frac{g'}{\sqrt{g}} \ll g$. This is what we mean by a "slowly varying" potential. Lets think about this physically. 

\begin{align}
g' &\sim \frac{g}{L}~,L = \text{length over which potential varies}\\
\frac{1}{L\sqrt{g}} &\ll 1\\
\frac{1}{\sqrt{g}} & \ll L\\
\phi &= \int \sqrt{g}dx\\
\phi &= \int \frac{dx}{\lambda}~,\text{where $\lambda$ is the wavelength to our solution on order $\frac{1}{\sqrt{g}}$}
\end{align}

Our WKB solution is okay if $\frac{1}{\sqrt{g}} \ll L$, $\lambda \ll L$. In our case, $\lambda$ is the deBr\"oglie wavelength. 

\begin{align}
g = \frac{2m_r}{\hbar^2} \left( E - \frac{e^2 Z_1 Z)_2}{r}\right)
\end{align}


\subsection{Tunneling}

\begin{align}
f(r_n) = e^{i \int_{r_n}^{r_c} \sqrt{g} dr} = e^{-\int_{r_n}^{r_c} \sqrt{|g|}dr}\\
P = e^{-I}~, I &= 2\int_{r_n}^{r_c} \sqrt{|g|}dr\\
I &= \frac{2\sqrt{2m_rE}}{\hbar} \int_{r_n}^{r_c} \left(  \frac{e^2 Z_1 Z_2}{r} -E \right)^{1/2} dr\\
& = \frac{2\sqrt{2m_rE}}{\hbar}    \int_{r_n}^{r_c} \left( \frac{r_c}{r} -1 \right)^{1/2}dr\\
x = \frac{r}{r_c}\\
I & = r_c \int_{r_n/r_c}^{x=r/r_rc} \left(\frac{1}{x}-1\right)^{1/2}dx\\
\int_0^1 \left(\frac{1}{x}-1\right)^{1/2}dx = \frac{\pi}{2}
\end{align}

Tunneling is independent of where nuclear reaction becomes important. Tunneling dominates at classical point. Once you get through the turning point, it doesn't matter how far you have to go. 

\begin{align}
I = \pi \sqrt{ \frac{2m_r e^4 Z_1^2 Z_2^2}{\hbar^2 E} } = \left(\frac{E_g}{E} \right)^{1/2}~, E_g &= \frac{2\pi^2 m_r e^4 Z_1^2 Z_2^2}{\hbar^2}\\
E &\sim E_g~, I \sim 1~,\text{Prob of tunneling}~ \sim 1\\
E& \ll E_g~, I \gg 1, P \ll 1\\
E_g = 1 ~\text{MeV} \frac{M_r}{m_p}Z_1^2 Z_2^2\\
P \approx e^{-(E_g/E)^{1/2}}
\end{align}
If $E$ is too low, no significant tunneling and no significant fusion.

At center of sun....

\begin{align}
T_{center} \sim 10^7 \text{ K} \sim 1 \text{ KeV}\\
\frac{3}{2}kT &= 2\text{ keV}\\
& \ll E_g \sim 500 \text{ keV}\\
P \sim 10^{-7}~,\text{damn, that's low.}
\end{align} 

Let's imagine particles with 10 times the thermal energy. $E = 10 E_{th} = 20\text{ keV}$, then $P \sim 10^{-2}$. This tells us that clearly particles which are more energetic that average are \textbf{MUCH} more likely to tunnel and thus undergo fusion. So when do we treat things quantum mechanically? If deBr\"oglie wavelength is large. Recap, $\lambda = \frac{h}{p} \sim \frac{h}{\sqrt{2Em_r}}$. As $E \uparrow$, $\lambda \downarrow$. Higher E has a smaller classical turning point. Then, $r_c \downarrow$ as $E \uparrow$. Yes, in absolute cm that the $r_c$ goes down, but relatively, it's easier to tunnel at higher $E$. 

\begin{align}
I &= (E_g/E)^{1/2}\\
& = \frac{\pi \sqrt{2m_rE}r_c}{\hbar} \sim \frac{r_c}{\lambda}
\end{align}

At high Z, $E_g \uparrow \rightarrow T \uparrow$. H is easiest to fuse at earliest stages.
