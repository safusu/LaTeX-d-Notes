\chapter{Understanding Stellar Evolution}

\begin{center}
\textbf{\begin{huge} October 13, 2011\end{huge}}
\end{center}

The balance we're interested in is:

\begin{align}
L_{fusion} &\approx L_{rad/conv}\\
	& + 	\text{HE}\rightarrow\text{main sequence}
\end{align}

For stars of $M \leq M_\odot$, supported by $P_{gas}$, pp chain fusion, and $\kappa \sim \kappa_{ff}$. And if $\gamma$ carry energy out, $L_{rad} \propto \frac{ M^{5.5} }{\sqrt{R}}$. Estimating $L_{fusion} = \epsilon_c M$. In the case of pp fusion, $\epsilon_{pp} \propto \rho T_c^{4/5}$ where $kT_c \sim \frac{GM \mu m_p}{R}$. \\

\begin{align}
L_{fusion} \pt \frac{M^{6/5}}{R^{7.5}}
\end{align}

If you change the $L$ of the star, $T$ and $\rho$ must change to accommodate. In steady state:

\begin{align}
L_{fusion} \pt \frac{M^{6/5}}{R^{7.5}} & \pt \frac{ M^{5.5} }{\sqrt{R}}\\
R & \pt M^{1/7}\\
T_c &\pt M^{6/7}\\
L & \pt M^{5.4}\\
L & \pt T_{eff}^4~ \pt 4\pi R^2\sigma T^4~, \text{but R dependence is so weak it's estimated to be constant}
\end{align}

\begin{align}
M \uparrow~,T_c \uparrow \sim M^{6/7}\\
\epsilon_{pp} \pt T^{4.5}~, \epsilon_{CNO} \pt T^{20}
\end{align}

Stars that are more massive than the sun are very $T$ dependent. \\

For $M \geq M_\odot$: $\kappa \sim \kappa_T \sim$ constant. $\gamma$ still dominate $E$ transport. CNO cycle is dominant mechanism and $P_{gas}$ dominates. $L_{rad} \pt M^3$.

\begin{align}
L_{fusion}&  = \int \epsilon_{CNO} dM_r\\
 & \sim \epsilon_{CNO}M~, \epsilon_{CNO} \pt \rho T_c^{20}\\
 \epsilon_{CNO} \pt \frac{M^{21}}{R^{23}}\\
 L_{fusion} \sim L_{rad}\\
  \frac{M^{22}}{R^{23}} \pt M^3\\
  R \pt M^{19/23} \pt M^{.8}\\
  T_c \pt M^{.2}
\end{align}

\begin{align}
L & = 4 \pi R^2 \sigma T_{eff}^4\\
L & \pt R^2 T_{eff}^4\\
R^2  & \pt M^{1.6} \pt L^{1/2}\\
L & \pt M^3\\
   & \pt L^{1/2}T_{eff}^4\\
L^{1/2} & \pt T_{eff}^4\\
L & \pt T_{eff}^8\\
T_{eff} & \pt M^{3/8}
\end{align}

A huge change in L corresponds to a small change in $T$. Only for stars with masses slightly more than the sun. \\

For $M \geq 1.2 M_\odot$, they have convective cores and $\gamma$ transport energy on outer part of star. Reverse of our sun. Convection sets in if $\frac{ds}{dr} < 0$. $\frac{ds}{dr}$ is implied by $\gamma$ transport of energy. You can then use radiative diffusion equation to see if $\frac{ds}{dr}<0$. i.e. Convection sets in if $\frac{d \ln T}{d \ln P} > \frac{\gamma -1}{\gamma}$. $\gamma$ is the one for photons, not of the particles convecting. 

\begin{align}
\frac{d \ln T}{d \ln P}  = \frac{1}{4} \frac{P}{P_{rad}}\frac{L}{L_{edd}}\frac{L_r/L}{M_r/M}~,L_{edd} = \frac{4 \pi G M_c}{\kappa}
\end{align}

$P$ is the total pressure. This tells us convection sets in if $\frac{1}{4} \frac{P}{P_{rad}}\frac{L}{L_{edd}}\frac{L_r/L}{M_r/M} > \frac{2}{5}$. Recall for CNO, $\epsilon \pt \rho T^\beta$. At almost all $r$, $L_r \approx L$. For CNO-dominated stars, only 1\% of star's mass fuses. TINY! It's this enormous flux that originates so close so the core that it drives convection. $\frac{M_r}{M} < \frac{5}{8} \frac{P}{P_{rad}}\frac{L}{L_{edd}}$, then convection sets in. We're interested where $P \approx P_{rad}$. We want to know $\frac{P}{P_{rad}}$ and $\frac{L}{L_{edd}}$. $L \pt M^3$ and $L_{EDD} \pt M$ so $\frac{L}{L_{edd}} = 4.5 \times 10^{-5} \left( \frac{M}{M_\odot} \right)^2$. \\

In the sun, $r \sim 0 \rightarrow .5 R_\odot$, $\frac{P_{gas}}{P_{rad}} \sim 3000$.

\begin{align}
P_{gas} &\pt \rho T \pt \frac{M^2}{R^4}\\
P_{rad}& = \frac{1}{3}a T^4 \pt \frac{M^4}{R^4}\\
\frac{P_{gas}}{P_{rad}} &\pt M^{-2}\\
\frac{P_{gas}}{P_{rad}} &\approx 3000 \left( \frac{M}{M_\odot} \right)^{-2}
\end{align}

Convection sets in if $\frac{M_r}{M} \leq 0.1$. \\

Lifetime of MS star $\approx \frac{E_{nuc}}{L}$. 

\begin{align}
L = L_\odot \left( \frac{M}{M_\odot} \right)^{3.5}\\
E_{nuc} &= N_p E\\
&  \approx  .1 \frac{M}{m_p}~\times 7 ~\text{MeV}\\
\frac{E_{nuc}}{L} \approx 10^{10} \left( \frac{M}{M_\odot} \right)^{-2.5}
\end{align}

ANY star with a mass less than .85$M_\odot$ is still fusing after 13.7 billion years. For $M \sim 30 M_\odot$, $t_{MS}$ is about $10^6$ years. Massive stars live and die where they are born. 