\chapter{Stellar Evolution}

\begin{center}
\textbf{\begin{huge} November 03, 2011\end{huge}}
\end{center}

\section{Stellar Evolution}

\begin{align}
t_{MS} = \frac{E_{NUC}}{L} \approx 10^{10} \mfrac^{-2.5} \textrm{ years}
\end{align}

On the Main Sequence, the star undergoes fusion from H into He in the core where then the properties of the star change in time. Radiative transport of energy dominates where $\sigma = \sigma_T$. $L \pt M^3 \mu^4 \mu_e$. The important thing to note is that on the MS, the luminosity of the star increases as it ages. \\

At time = 0, $X = .75, Y = .25, \mu = 0.6, \mu_e=1$ which then turns into pure He, $\mu_e=2, \mu = \sfrac{4}{3} \ra \frac{L(t_{MS})}{L(t=0)} \approx 40$, which arises just from the change in composition of the star. But we know the entire star doesn't convert into He, except for fully convective star where there's a lot of mixing. To do this correctly...\\
$0.1 M:$ H $\ra$ He (core) with properties: $X = 0.65, Y = 0.35, \mu = 0.64, \frac{L(t_{MS})}{L(t=0)} = 1.4$, which means at the end of its life cycle, it'll be 40\% brighter. For us, this corresponds to a change in $T$ of about 60$^{\circ}$F. This gives rise to the issue that earlier in the sun's lifetime it was significantly cooler and thus the earth might have had problems evolving life. Buuuut this is stellar physics not astrobiology or whatever. Why Fahrenheit all of a sudden? Don't worry about it.\\

Let's start with a nice picture of a star with a He core and a H shell where the mass in the He core is about 0.1$M_{tot}$. Since this core isn't fusing, it's undergoing KH contraction. Core contraction causes $T$ to go up, causing fusion of heavier elements which then runs out, contracts, ad infinitum (At least until Fe or something). This cycle involves simultaneously the $T_c$ and $\rho_c$ both going up. What stops this cycle? It depends on two pieces of key physics:\\

\begin{list}{$\circ$}{}
\item An object supported by electron degeneracy pressure has a maximum mass (Chandrasekhar mass, about 1.4 $\ms$). We'll prove this later. 
\end{list}

\noindent If the mass of the star's core is greater than $M_{ch}$, then the degeneracy pressure can't stop contraction. 

\begin{list}{$\circ$}{}
\item $T_{max} \approx 8 \times 10^7$ K $\mfrac^{4/3}$ of a star with gas pressure and NR degeneracy pressure.
\end{list}

\noindent i.e. if a star is supported by NR degeneracy pressure, you can't be hotter than the $T$ above. Imposing the two conditions, then you get:

\begin{align}
M < M_{ch} \ra T_{max} < 2 \times 10^8\textrm{ K}~.
\end{align}

\noindent This means you can't fuse to arbitrarily high Z. In particular, things that have $M<M_{ch}$ can only fuse He $\ra$ C,O but no higher. To fuse Mg, you need $T \approx 10^9$ K or so. 

\section{What is the End Fate of a Star?}

... and it's dependency on it's initial mass?\\

We have to take into account the difference of mass between final mass and initial mass. Towards the end of a star's life (esp. massive stars), a lot of mass is lost. Roughly, stars with $M_i \lesssim 8 \ms$, they end up with a core with a mass less than $M_{ch}$, becoming supported by electron degeneracy, which stops the cycle of KH contraction. These are usually C/O composition. This is a white dwarf, in particular, a C/O white dwarf. This is exactly the same argument as to why brown dwarves exist. The change of mass, $\sim 6\ms$, is blown away by powerful stellar winds. Some dwarves continue to fuse, but most stop at C/O. Stars with $M < 0.5 \ms$, you never fuse He. The MS lifetime of a $0.5 \ms$ is longer than the age of the universe... but we still see them anyways. 

\subsection{$M_i > 8\ms$}

These end with a core heavier than $M_{ch}$ so they can't be supported by electron degeneracy. This also means there is no maximum temperature. Using V.T and some other stuff, $T \pt \rho^{1/3}$  and if one increases, so does the other. Eventually you fuse all the heavier elements up to iron. Iron is important because they're the tightest bound atomic nuclei. Once you hit Iron, fusion is no longer a source of energy. Now, there's nothing to maintain the $T$. It's too massive to be supported by electron degeneracy... leaving $P_{gas}$ and $P_{rad}$, but they both depend on $T$. You run into the case where you have no pressure support and the star collapses. The collapse produces either a neutron star or a black hole and an explosion. The core may be a NS or BH, but the explosion blows away the outer part of the star. This is a supernova. FYI, the in falling material is moving $\sim c$. Stellar evolution is a little complicated in that you have to keep separate what the core is doing and what the outer part of the star is doing. 

\section{End of the MS}

He core, surrounded mostly by H and other stuff. Imagine H fusion shuts off (in the center) which make the star contracts. This makes the $\rho_c,T_c \uparrow$ as well as the H envelope $\rho,T \uparrow$. You end up with a He core and a H fusing shell surrounded by the rest of the envelope. The core has to get pretty hot for it to fuse the next step but the shell only has to get a little hotter to fuse. The core is still contracting, resulting in the $L_{shell} \uparrow$. Out in the envelope, photons are carrying the energy, carrying out the $L_{rad} = L_\odot \mfrac^3$. At some point, this $L_{shell}$ is greater than the energy photons can carry out in the envelope. Now, you're going to start pushing stuff out, driving convection. Once the entire envelope becomes convective, it can carry energy out pretty effectively.\\

Now the star is a giant with the outer parts of the star convecting. The image here is a He core, H shell fusion, and big puffy convective envelope. On the MS, what happens if there was a little extra fusion in the center of the sun? It would expand and $T \downarrow$, but that doesn't happen here. In that scenario, the core and envelope are doing the same thing, which isn't happening here. Basically, the core is forcing the physics of the entire star. There's nothing stopping the core contracting (runaway) and there's no ``safety valve" of the MS. This contraction for the sun is extremely fast (on stellar scales) on about a $t_{KH}$ time scale $\sim 10^7$ years. If you look on the HR diagram, you'll see a lot of stars at the giant stage, then it's pretty empty until you hit the white dwarf area. An interesting note is that during the giant stage the $T_{eff}$ gets to about 3000-4000 K and scrapes the Hayashi Line.\\

Eventually, the core stops contracting and you can either have He Fusion or be supported by degeneracy pressure. Let's continue imagining a core undergoing He fusion and H fusion in shell where the $L_{shell}$ continues to increase in time. As H in the shell goes to He, the mass of the He core increases, causing the $T_{shell}$ to go up, causing $L_{shell}$ to increase. At this stage, the star is moving up the Hayashi Line (i.e. ``up the giant branch"). The reason the $L \uparrow$ is thanks to convection. As $L = 4 \pi R^2T_{eff}^2$ increases with $T_{eff}$ staying roughly constant, the radius goes up. Eventually, a bunch of H is fused and then He fusions begins. A star steadily fusing He has a relatively long lifetime; it's like a He MS. Lower mass stars have degenerate cores and undergoes degenerate He fusion. In this scenario, there is no ``safety valve" and with deg. He fusion, $P \pt \rho^{5/3}$, leading to runaway He fusion (He flash) puffing up the star quickly. \\

Any stars $0.5 \ms$ and higher go through the He fusion stage pretty predictably. Going into this later. 
